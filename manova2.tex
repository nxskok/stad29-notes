\documentclass{article}
\title{The manova2 data}
\usepackage{graphicx}
\usepackage{Statweave}
\begin{document}
\maketitle
The data:
\begin{verbatim}
low 34 10 
low 29 14 
low 35 11 
low 32 13 
high 33 14 
high 38 12 
high 34 13 
high 35 14 
\end{verbatim}
The SAS code and output:
\begin{Winput}
data manova2;
  infile "manova2.dat";
  input fertilizer $ yield weight;

proc discrim can list out=fred;
  class fertilizer;
  var yield weight;

proc print data=fred;

run;
\end{Winput}
\begin{Woutput}
The DISCRIM Procedure
Total Sample Size        8          DF Total                 7
Variables                2          DF Within Classes        6
Classes                  2          DF Between Classes       1

Number of Observations Read              8
Number of Observations Used              8

                           Class Level Information
              Variable                                                  Prior
fertilizer    Name        Frequency       Weight    Proportion    Probability
high          high                4       4.0000      0.500000       0.500000
low           low                 4       4.0000      0.500000       0.500000

Pooled Covariance Matrix Information
               Natural Log of the
 Covariance    Determinant of the
Matrix Rank     Covariance Matrix
          2               1.22255

The DISCRIM Procedure
Pairwise Generalized Squared Distances Between Groups
 2         _   _       -1  _   _
D (i|j) = (X - X )' COV   (X - X )
            i   j           i   j

Generalized Squared Distance to fertilizer
From
fertilizer          high           low
high                   0      12.11656
low             12.11656             0

The DISCRIM Procedure
Canonical Discriminant Analysis
                           Adjusted    Approximate        Squared
           Canonical      Canonical       Standard      Canonical
         Correlation    Correlation          Error    Correlation
       1    0.895289       0.892147       0.075010       0.801542
                                                      Test of H0: The canonical correlations in the
                   Eigenvalues of Inv(E)*H               current row and all that follow are zero
                     = CanRsq/(1-CanRsq)
                                                     Likelihood Approximate
         Eigenvalue Difference Proportion Cumulative      Ratio     F Value Num DF Den DF Pr > F
       1     4.0389                1.0000     1.0000 0.19845779       10.10      2      5 0.0175
NOTE: The F statistic is exact.

The DISCRIM Procedure
Canonical Discriminant Analysis
Total Canonical Structure
Variable              Can1
yield             0.572987
weight            0.495570

Between Canonical Structure
Variable              Can1
yield             1.000000
weight            1.000000

Pooled Within Canonical Structure
Variable              Can1
yield             0.297366
weight            0.246343

The DISCRIM Procedure
Canonical Discriminant Analysis
Total-Sample Standardized Canonical Coefficients
Variable              Can1
yield          1.997145424
weight         1.884468331

Pooled Within-Class Standardized Canonical Coefficients
Variable              Can1
yield          1.851698615
weight         1.824149648

Raw Canonical Coefficients
Variable              Can1
yield          0.766676064
weight         1.251356335

Class Means on Canonical Variables
fertilizer              Can1
high             1.740442790
low             -1.740442790

The DISCRIM Procedure
Linear Discriminant Function
               _     -1 _                              -1 _
Constant = -.5 X' COV   X      Coefficient Vector = COV   X
                j        j                                 j

Linear Discriminant Function for fertilizer
Variable          high           low
Constant    -943.76534    -798.70399
yield         33.60736      30.93865
weight        53.68098      49.32515

The DISCRIM Procedure
Classification Results for Calibration Data: WORK.MANOVA2
Resubstitution Results using Linear Discriminant Function
Generalized Squared Distance Function
 2         _       -1   _
D (X) = (X-X )' COV  (X-X )
 j          j            j
Posterior Probability of Membership in Each fertilizer
                   2                    2
Pr(j|X) = exp(-.5 D (X)) / SUM exp(-.5 D (X))
                   j        k           k

 Posterior Probability of Membership in fertilizer
                     Classified
       From          into
Obs    fertilizer    fertilizer      high       low
  1    low           low           0.0000    1.0000
  2    low           low           0.0012    0.9988
  3    low           low           0.0232    0.9768
  4    low           low           0.0458    0.9542
  5    high          high          0.9818    0.0182
  6    high          high          0.9998    0.0002
  7    high          high          0.9089    0.0911
  8    high          high          0.9999    0.0001

The DISCRIM Procedure
Classification Summary for Calibration Data: WORK.MANOVA2
Resubstitution Summary using Linear Discriminant Function
Generalized Squared Distance Function
 2         _       -1   _
D (X) = (X-X )' COV  (X-X )
 j          j            j
Posterior Probability of Membership in Each fertilizer
                   2                    2
Pr(j|X) = exp(-.5 D (X)) / SUM exp(-.5 D (X))
                   j        k           k

Number of Observations and Percent Classified into fertilizer
From
fertilizer         high          low        Total
high                  4            0            4
                 100.00         0.00       100.00
low                   0            4            4
                   0.00       100.00       100.00
Total                 4            4            8
                  50.00        50.00       100.00
Priors              0.5          0.5

      Error Count Estimates for fertilizer
                    high         low       Total
Rate              0.0000      0.0000      0.0000
Priors            0.5000      0.5000

Obs    fertilizer    yield    weight      Can1      Can2      high       low      _INTO_
 1        low          34       10      -3.09314      .     0.00002    0.99998     low
 2        low          29       14      -1.92110      .     0.00125    0.99875     low
 3        low          35       11      -1.07511      .     0.02315    0.97685     low
 4        low          32       13      -0.87242      .     0.04579    0.95421     low
 5        high         33       14       1.14561      .     0.98180    0.01820     high
 6        high         38       12       2.47628      .     0.99982    0.00018     high
 7        high         34       13       0.66093      .     0.90893    0.09107     high
 8        high         35       14       2.67896      .     0.99991    0.00009     high
\end{Woutput}
\end{document}
