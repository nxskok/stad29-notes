
\part{Chapter 9: Analysis of two-way tables}

\begin{slide}{Comparing multiple proportions (\S 9.1)}
  \begin{itemize}
  \item 
We have been stressing the need to do {\em comparative experiments}
where possible. For instance: in type A people who have survived heart
attacks, is it helpful to offer behavioral training as well as medical
care?

\item 290 patients randomized into 2 groups. Treatment group got behavioral training
plus medical care, while control group got medical care only. Each person
either suffered a 2nd heart attack or not.

\item For two proportions, can use methods of \S 8.2; for more, need
  something new. (Compare this example using two methods.)

  \end{itemize}
\end{slide}
\begin{slide}{Contingency table}
  \begin{itemize}
  \item 
Summarize results this way:

\begin{tabular}{l|cc|c}
\hline
& 2nd attack & No 2nd attack & Total\\
\hline
Treatment & 17 & 123 & 140\\
Control   & 29 & 121 & 150\\
\hline
Total & 46 & 244 & 290\\
\hline
\end{tabular}

\item For instance, 123 patients were in treatment group and did not get a
2nd heart attack. Layout called {\bf contingency table}.

\item Research question: does treatment affect rate of 2nd heart attacks?

  \end{itemize}
\end{slide}
\begin{slide}{Percentages}
  \begin{itemize}
  \item 
Starting point: percentages. Depending on data, row/column/overall
\%'s may be best. Here, want \% of 2nd attacks for treatment \&
control groups.

\item Treatment: $17/140\times 100\% = 12\%$, control: $29/150\times
100\%=19\%$.  Fewer 2nd attacks in treatment group -- but could be
just chance.

  \end{itemize}
\end{slide}

\begin{slide}{Inference for two or more proportions (\S 9.2)}
  \begin{itemize}
  \item 
Alternative hypothesis: treatment has some effect (treatment
proportion of 2nd attacks different from control proportion). Null
hyp.: proportions same.

\item In example, expect more 2nd attacks in control group because 150
patients vs.\ 140. This true even if treatment has no effect. But how
many? 

\item Idea: 46 of 290 patients (0.1586) had 2nd attack overall. So, if
null hyp.\ true, expect $0.1586\times 140=22.21$ in treatment group,
$0.1586 \times 150=23.79$ in control group.

  \end{itemize}
\end{slide}
\begin{slide}{Expected counts}

Work out expected numbers of patients without 2nd attacks in same way,
get this:

\scriptsize
\begin{verbatim}
Expected counts are printed below observed counts

            C1       C2    Total
    1       17      123      140
         22.21   117.79

    2       29      121      150
         23.79   126.21

Total       46      244      290
\end{verbatim}
\normalsize

\end{slide}
\begin{slide}{Chisquare test}

Now want one number summarizing this: small if observed and expected all
close, large otherwise. Right mathematics p.\ 610; use Minitab
calculation:

\scriptsize
\begin{verbatim}
Chi-Sq =  1.221 +  0.230 +
          1.139 +  0.215 = 2.805
DF = 1, P-Value = 0.094
\end{verbatim}
\normalsize

At $\alpha=0.05$, can't reject null hyp. Evidence in data set not
strong enough to conclude that treatment has effect.

Procedure called a {\bf chisquare test}. 

\end{slide}

\begin{slide}{Doing it all in Minitab}
  \begin{itemize}
  \item 
Calculations above were actually Minitab output. First step is to get
table into Minitab. Just enter counts as laid out in table (without
totals): 17 and 29 in column C1, 123 and 121 in column C2.

\item Then: Stat, Tables, Chisquare Test. Select C1 and C2 as columns
containing table; get output as above.

  \end{itemize}
\end{slide}

\begin{slide}{Another example}

Altruism defined as ``interest in welfare of others''. Questionnaire
developed to measure altruism -- results low, medium, high. Students
from different majors studied:

\begin{tabular}{lcccc}
\hline
Major & Low & Medium & High\\
\hline
Agriculture & 5 & 27 & 35\\
Family studies & 1 & 32 & 34\\
Engineering & 12 & 129 & 94\\
Education & 7 & 77 & 129\\
Management & 3 & 44 & 28\\
Science & 7 & 29 & 24\\
Technology & 2 & 62 & 64\\
\hline
\end{tabular}

Comparing many proportions. Analysis from Minitab (tidied):

\end{slide}
\begin{slide}{Expected counts}

\footnotesize
\begin{verbatim}
Expected counts are printed below observed counts

           Low   Medium     High    Total
Agric        5       27       35       67
          2.87    30.98    33.15
Family       1       32       54       87
          3.72    40.23    43.05
Engin       12      129       94      235
         10.05   108.67   116.28
Educat       7       77      129      213
          9.11    98.50   105.39
Manage       3       44       28       75
          3.21    34.68    37.11
Science      7       29       24       60
          2.57    27.75    29.69
Techno       2       62       64      128
          5.48    59.19    63.33
Total       37      400      428      865
\end{verbatim}
\end{slide}
\begin{slide}{Chisquare test}

\begin{verbatim}
Chi-Sq =  1.589 +  0.512 +  0.103 +
          1.990 +  1.684 +  2.787 +
          0.377 +  3.803 +  4.268 +
          0.489 +  4.692 +  5.288 +
          0.013 +  2.503 +  2.236 +
          7.659 +  0.057 +  1.090 +
          2.206 +  0.133 +  0.007 = 43.487
DF = 12, P-Value = 0.000
4 cells with expected counts less than 5.0
\end{verbatim}
\normalsize

\begin{itemize}
\item 
P-value is small; reject null hyp. Definitely difference in
proportions of low, medium, high among majors.

\item How do majors differ? Compare observed and expected; something
interesting happening where very different:

\end{itemize}
\end{slide}
\begin{slide}{Why was $H_0$ rejected?}

\begin{itemize}
\item Science (row 6): more lows than expected.
\item Education (row 4): more highs, fewer medium/low than expected.
\item Engineering (row 3): fewer highs, more medium/low than expected.
\end{itemize}
According to questionnaire, education students more altruistic than
average, science and engineering students less so.

\end{slide}

\begin{slide}{Summary}

\begin{itemize}
\item Often do comparative studies of two or more groups, yes/no type answers.
\item Contingency table to display results; percents to summarize.
\item Calculate expected frequencies;
leads to chi-square test: null of all proportions same,
  alternative not all same. Reject: conclude some differences among
  proportions.
\end{itemize}
\end{slide}
%%% Local Variables: 
%%% mode: latex
%%% TeX-master: "slides"
%%% End: 
