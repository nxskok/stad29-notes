\section{Multivariate ANOVA}

\documentclass[pdf]{prosper}
\usepackage[Lakar]{HA-prosper}
\usepackage{graphicx}

\begin{document}

\begin{slide}{Multivariate analysis of variance}

  \begin{itemize}
  \item Standard ANOVA has just one response variable.
  \item What if you have more than one response?
  \item Try an ANOVA on each response separately.
  \item But might miss some kinds of interesting dependence between the responses that distinguish the groups.
  \item SAS can run MANOVA using an option on PROC GLM.
  \end{itemize}
  
\end{slide}

\begin{slide}{Small example}

  \begin{itemize}
  \item Measure yield and seed weight of plants grown under 2 conditions: low and high amounts of fertilizer.
  \item Data (fertilizer, yield, seed weight):
\begin{verbatim}
low 34 10
low 29 14
low 35 11
low 32 13
high 33 14
high 38 12
high 34 13
high 35 14

\end{verbatim}
  \item 2 responses, yield and seed weight.
  \item First get means by fertilizer amount.
  \item Then run 1-way ANOVA for each of yield and seed weight, using fertilizer type as explanatory.
  \end{itemize}
  
\end{slide}

\begin{slide}{Code}

\begin{verbatim}
data manova1;
  infile "manova1.dat";
  input fertilizer $ yield weight;

proc means;
  var yield weight;
  class fertilizer;

proc glm;
  class fertilizer;
  model yield=fertilizer;

proc glm;
  class fertilizer;
  model weight=fertilizer;

\end{verbatim}
  
\end{slide}

\begin{slide}{The means}

{\scriptsize
\begin{verbatim}
                          The MEANS Procedure

                N
  fertilizer  Obs  Variable  N          Mean       Std Dev       Minimum
  ----------------------------------------------------------------------
  high          4  yield     4    35.0000000     2.1602469    33.0000000
                   weight    4    13.2500000     0.9574271    12.0000000

  low           4  yield     4    32.5000000     2.6457513    29.0000000
                   weight    4    12.0000000     1.8257419    10.0000000
  ----------------------------------------------------------------------

\end{verbatim}
}

Means on both variables are slightly higher for high fertilizer. Are those differences significant? Look at ANOVAs (2-sample $t$-tests would also have worked.)
  
\end{slide}

\begin{slide}{The ANOVAs}

Only one $x$ (fertilizer amount) so look at ``model'' line.

{\scriptsize
\begin{verbatim}
Dependent Variable: yield   
                                     Sum of
 Source                    DF       Squares   Mean Square  F Value  Pr > F
 Model                      1   12.50000000   12.50000000     2.14  0.1936
 Error                      6   35.00000000    5.83333333                 
 Corrected Total            7   47.50000000                               

Dependent Variable: weight   
                                     Sum of
 Source                    DF       Squares   Mean Square  F Value  Pr > F
 Model                      1    3.12500000    3.12500000     1.47  0.2708
 Error                      6   12.75000000    2.12500000                 
 Corrected Total            7   15.87500000                               
\end{verbatim}
}

Neither mean yield nor mean weight depends on the amount of fertilizer. But: look at plot of yield vs.\ weight labelled by fertilizer, using this code:

\begin{verbatim}
proc plot;
  plot yield*weight $ fertilizer;
\end{verbatim}
  
\end{slide}

\begin{slide}{Plot of yield vs.\ weight}

{\scriptsize
\begin{verbatim}
         Plot of yield*weight$fertilizer.  Symbol points to label.
     yield
        38 +                            > high
           |
        37 +
           |
        36 +
           |
        35 +               > low                             high <
           |
        34 +  > low                                  > high
           |
        33 +                                                 high <
           |
        32 +                                         > low
           |
        31 +
           |
        30 +
           |
        29 +                                                  low <
           ---+------------+------------+------------+------------+--
             10           11           12           13           14
                                     weight

\end{verbatim}
}
  
\end{slide}

\begin{slide}{MANOVA code}
  \begin{itemize}
  \item High-fertilizer plants have both yield and weight high.
  \item True even though no sig difference in yield or weight individually.
  \item Could draw a line separating highs from lows on graph.
  \item Is {\em that} significant? MANOVA finds out.
  \item Code:
  \end{itemize}

\begin{verbatim}
proc glm;
  class fertilizer;
  model yield weight=fertilizer;
  manova h=_all_;

\end{verbatim}

\end{slide}

\begin{slide}{Output}

Includes this:

{\scriptsize
\begin{verbatim}
                             The GLM Procedure
                     Multivariate Analysis of Variance

Statistic                       Value   F Value   Num DF   Den DF   Pr > F
Wilks' Lambda              0.19845779     10.10        2        5   0.0175
Pillai's Trace             0.80154221     10.10        2        5   0.0175
Hotelling-Lawley Trace     4.03885481     10.10        2        5   0.0175
Roy's Greatest Root        4.03885481     10.10        2        5   0.0175

\end{verbatim}
}

\begin{itemize}
\item 
Four versions of ANOVA $F$-test, here all agree: the multivariate difference seen on graph {\em is} significant. 
\item With more than 2 responses, cannot draw graph. What then? 
\item Use {\em discriminant analysis} (of which more later).
\end{itemize}
  
\end{slide}

\begin{slide}{A discriminant analysis}

Treat this as ``magic'' for now, but: obtain output data set and look at it.

\begin{verbatim}
proc discrim can out=fred;
  class fertilizer;
  var yield weight;

proc print data=fred;

\end{verbatim}

Ignore most of output from PROC DISCRIM, then look at output data set.

\end{slide}

\begin{slide}{Output}

{\scriptsize
\begin{verbatim}
                Linear Discriminant Function for fertilizer
 
                   Variable          high           low

                   Constant    -943.76534    -798.70399
                   yield         33.60736      30.93865
                   weight        53.68098      49.32515
\end{verbatim}
}

\begin{itemize}
\item For an observation's yield and weight, calculate discriminant functions, classify into fertilizer group with higher one.
\item SAS does this for us (see Can1 below).
\end{itemize}

\end{slide}

\begin{slide}{Output data set}

{\scriptsize
\begin{verbatim}
 Obs  fertilizer  yield  weight    Can1    Can2    high     low    _INTO_
  1      low        34     10    -3.09314    .   0.00002  0.99998   low  
  2      low        29     14    -1.92110    .   0.00125  0.99875   low  
  3      low        35     11    -1.07511    .   0.02315  0.97685   low  
  4      low        32     13    -0.87242    .   0.04579  0.95421   low  
  5      high       33     14     1.14561    .   0.98180  0.01820   high 
  6      high       38     12     2.47628    .   0.99982  0.00018   high 
  7      high       34     13     0.66093    .   0.90893  0.09107   high 
  8      high       35     14     2.67896    .   0.99991  0.00009   high 
\end{verbatim}
}

\begin{itemize}
\item In \verb-Can1-,
 low value suggests low fertilizer, high suggests high.
\item ``high'' and ``low'' are estimated probabilities that observation with that yield and weight was high or low fertilizer.
\item Last column is SAS's guess at which group it comes from (higher est prob). Got them all right.
\item Distinction between high and low quite clear when looked at the right way.
\item Procedure works no matter what combination of responses best divides data into groups by $x$. 
\end{itemize}
  
\end{slide}

\end{document}
