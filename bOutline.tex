\section*{Course Outline}
\frame{\sectionpage}

\begin{frame}[fragile]{Course and instructor}
  \begin{itemize}
    \item  Lecture: Wednesday 14:00-16:00 in HW 215. Optional computer
      lab Monday 16:00-17:00 in BV 498.
    \item  Instructor: Ken Butler
    \item  Office: IC 471.
    \item  E-mail: \verb-butler@utsc.utoronto.ca-
    \item Office hours: Monday 11:00-13:00. Also, Wednesday mornings
      good. I am often around. See if I'm in. Or make an
      appointment. E-mail always good.
    \item Course website: 
\url{www.utsc.utoronto.ca/~butler/d29}.
    \item Using Blackboard for assignments/grades only; using website for
      everything else.
\end{itemize}

\end{frame}

\begin{frame}[fragile]{Text, programs, prerequisites and exclusions}

\begin{itemize}
\item There is no official text for this course. If you want something
  to refer to, there is
      \verb=www.utsc.utoronto.ca/~butler/r/r-howto.pdf=. 
      ``The book'', free for the download. This is a bit outdated now.
    \item Prerequisites:
      \begin{itemize}
      \item For undergrads: STAC32. Not negotiable.
    \item  For grad students,
      a first course in statistics, and some training in
      regression and ANOVA. The less you know, the more you'll have to
      catch up!
      \end{itemize}

    \item This course is part of Applied Statistics minor.
    \item Exclusions: \textbf{this course is not for Math/Statistics/CS majors/minors}. It
      is for
      students in other fields who wish to learn some more advanced
      statistical methods. The exclusions in the Calendar reflect
      this. (If you are in one of those programs, you won't get
      program credit for this course.)
\end{itemize}
  
\end{frame}


\begin{frame}[fragile]{Computing}

  \begin{itemize}
  \item Computing: big part of the course, {\bf not}
    optional. Demonstrate that you can use 
    R to analyze data, and can
    critically interpret the output.
  \item A brief introduction will be available
    for those who have not come through STAC32. I am happy to offer
    extra help to these people.
  \end{itemize}
  
\end{frame}


\begin{frame}{Computing and assessment}
\begin{itemize}
\item Grading: (2 hour) midterm, (3 hour) final exam. Assignments most
  weeks, due Tuesday at 11:59pm. 
  Graduate students (STA 1007) also required to
  complete a project using one or more of the techniques learned in
  class, on a dataset from their field of study.    Projects due on
  the last day of classes.

\item Assessment:

  \begin{tabular}{lcc}
    & STAD29 & STA 1007\\
    Assignments & 20\% & 20\%\\
    Midterm exam & 30\%  & 20\% \\
    Project & - & 20\%\\
    Final exam & 50\% & 40\%
  \end{tabular}

\item Assessments missed \emph{with documentation} will cause a
  re-weighting of other assessments of same type. No make-ups.
\item You \textbf{must pass the final exam} to pass the course. If you
  fail the final exam but would otherwise have passed the course, you
  receive a grade of 45.

\end{itemize}
\end{frame}

\begin{frame}{Plagiarism}

  \begin{itemize}
  \item
    \url{http://www.utoronto.ca/academicintegrity/academicoffenses.html}
    defines academic offences at this university. Read it.
  \item Plagiarism is defined (at the end) as
    \begin{quote}
       The wrongful appropriation and purloining, and publication as one’s own, of the ideas, or the expression of the ideas ... of another.
    \end{quote}
    \item The code and
    explanations  that
    you write and hand in must be \emph{yours and yours
      alone}. 
    \item When you hand in work, it is implied that it is
    \emph{your} work. Handing in work, with your name on it, that was actually done by
    someone else is an \emph{academic offence}.
  \item If I am suspicious
    that anyone's work is plagiarized, I will take action.
    
  \end{itemize}
  
\end{frame}

\begin{frame}[fragile]{Getting help}

  \begin{itemize}
  \item The English Language Development Centre supports all students
    in developing better Academic English and critical thinking skills
    needed in academic communication. Make use of the personalized
    support in academic writing skills development. Details and sign-up information:
    \url{http://www.utsc.utoronto.ca/eld/}.
  \item      Students with diverse learning styles and needs are welcome in this
course. In particular, if you have a disability/health consideration
that may require accommodations, please feel free to approach the AccessAbility Services Office as soon as possible. I will
work with you and AccessAbility Services to ensure you can achieve
your learning goals in this course. Enquiries are confidential. The
UTSC AccessAbility Services staff are available by
appointment to assess specific needs, provide referrals and arrange
appropriate accommodations: (416) 287-7560 or by e-mail: \texttt{ability@utsc.utoronto.ca}.


  \end{itemize}
  
\end{frame}
