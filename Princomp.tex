\documentclass[pdf]{prosper}
\usepackage[Lakar]{HA-prosper}
\usepackage{graphicx}

\begin{document}

\begin{slide}{Principal Components}
  \begin{itemize}
  \item Have measurements on (possibly large) number of variables on some individuals.
  \item Question: can we describe data using fewer variables (because original variables correlated in some way)?
  \item Look for direction (linear combination of original variables) in which values {\em most spread out}. This is {\em first principal component}.
  \item Second principal component then direction uncorrelated with this in which values then most spread out. And so on.
  \item See whether small number of principal components captures most of variation in data.
  \item Might try to interpret principal components.
  \item If 2 components good, can make plot of data.
  \item (Akin to ideas in discriminant/canonical variables analysis, but no groups here.)
\end{itemize}

\end{slide}

\begin{slide}{Small example: 2 test scores for 8 people}

{\scriptsize
\begin{verbatim}
score1 |
    18 +                                                    A
       |
    16 +                                                A
       |
    14 +
       |
    12 +                                    A
       |
    10 +                              A A
       |
     8 +                    A
       |
     6 +
       |
     4 +            A
       |
     2 +           A
       |
       -+-----------+-----------+-----------+-----------+-----------+-
        0          10          20          30          40          50

                                    score2

\end{verbatim}
}

  
\end{slide}

\begin{slide}{Principal component analysis}

Strongly correlated, so data nearly 1-dimensional. Make a score summarizing this one dimension.


Code like this:

\begin{verbatim}
options linesize=70;

data test;
  infile "test12.dat";
  input score1 score2;

proc princomp out=fred;

proc print;

\end{verbatim}
  
\end{slide}

\begin{slide}{The output}

{\scriptsize
\begin{verbatim}
                          Correlation Matrix
 
                                score1      score2
                    score1      1.0000      0.9891
                    score2      0.9891      1.0000

                 Eigenvalues of the Correlation Matrix
 
             Eigenvalue    Difference    Proportion    Cumulative
        1    1.98907796    1.97815591        0.9945        0.9945
        2    0.01092204                      0.0055        1.0000
\end{verbatim}
}

\begin{itemize}
\item The two variables are very highly correlated.
\item Look at {\em eigenvalues}:
  \begin{itemize}
  \item First one is much bigger than rest, so data ``almost'' 1-dimensional.
  \item Last column: first principal component accounts for almost all (99.45\%) of variability in data, so we do fine by summarizing data by 1st principal component.
  \item Generally: consider retaining components with eigenvalues bigger than 1.
  \end{itemize}
\end{itemize}
  
\end{slide}

\begin{slide}{Eigenvectors}

{\scriptsize
\begin{verbatim}
                            Eigenvectors
 
                                 Prin1         Prin2

                  score1      0.707107      0.707107
                  score2      0.707107      -.707107
\end{verbatim}
}

\begin{itemize}
\item Eigenvectors show how principal components depend on original variables (standardized).
\item 1st principal component is basically sum of \verb-score1- and \verb-score2-, standardized.
\item If correlation between 2 test scores had been negative, 1st eigenvector would have said ``look at difference''.
\end{itemize}
\end{slide}

\begin{slide}{Output data set}

{\scriptsize
\begin{verbatim}

           Obs    score1    score2      Prin1       Prin2

            1        2         9      -1.93801    -0.13749
            2       16        40       1.60878    -0.05216
            3        8        17      -0.71306     0.19418
            4       18        43       2.03571     0.03979
            5       10        25      -0.00698     0.00698
            6        4        10      -1.62274     0.06612
            7       10        27       0.10468    -0.10468
            8       12        30       0.53161    -0.01273

\end{verbatim}
}

\begin{itemize}
\item Values on \verb-Prin1- identify each person as a ``high scorer'' (positive) or ``low scorer'' (negative).
\item \verb-Prin1- and \verb-Prin2- called {\em principal
    component scores}.
\end{itemize}
  
\end{slide}

\begin{slide}{Track running data}

(1984) track running records for distances 100m to marathon, arranged by country. Countries labelled by (mostly) Internet domain names --- actual names not in file. 

{\scriptsize
\begin{verbatim}
  10.39  20.81  46.84  1.81  3.70  14.04  29.36  137.72  ar (Argentina)
  10.31  20.06  44.84  1.74  3.57  13.28  27.66  128.30  au (Australia)
  10.44  20.81  46.82  1.79  3.60  13.26  27.72  135.90  at (Austria)
  10.34  20.68  45.04  1.73  3.60  13.22  27.45  129.95  be (Belgium)
  10.28  20.58  45.91  1.80  3.75  14.68  30.55  146.62  bm (Bermuda)
  10.22  20.43  45.21  1.73  3.66  13.62  28.62  133.13  br (Brazil)
  10.64  21.52  48.30  1.80  3.85  14.45  30.28  139.95  bu (Burma)
  10.17  20.22  45.68  1.76  3.63  13.55  28.09  130.15  ca (Canada)
  10.34  20.80  46.20  1.79  3.71  13.61  29.30  134.03  cl (Chile)
  ....
  10.71  21.43  47.60  1.79  3.67  13.56  28.58  131.50  tr (Turkey)
   9.93  19.75  43.86  1.73  3.53  13.20  27.43  128.22  us (United States)
  10.07  20.00  44.60  1.75  3.59  13.20  27.53  130.55  ru (USSR)
  10.82  21.86  49.00  2.02  4.24  16.28  34.71  161.83  ws (Western Samoa)
\end{verbatim}
}
  
\end{slide}

\begin{slide}{Data and aims}

  \begin{itemize}
  \item 
Times in seconds 100m-400m, in minutes for rest (800m, 1500m, 5000m, 10000m, marathon).
\item This taken care of by (automatic) standardization.
\item 8 variables; can we summarize by fewer and gain some insight?
\item In particular, if 2 components tell most of story, what do we see in a plot?

  \end{itemize}

  
\end{slide}


\begin{slide}{Code}

\begin{verbatim}
options linesize=70;

data track;
  infile "men_track_field.dat";
  input m100 m200 m400 m800 m1500 m5000 m10000 marathon 
    country $;

proc princomp out=PC;

proc print data=PC;
  var country Prin1 Prin2;

proc plot data=PC;
  plot Prin2*Prin1 = '*' $ country;

\end{verbatim}
  
\end{slide}

\begin{slide}{Correlation matrix}

{\scriptsize
\begin{verbatim}

                               Correlation Matrix

               m100    m200    m400    m800   m1500   m5000  m10000  marathon

   m100      1.0000  0.9226  0.8411  0.7560  0.7002  0.6195  0.6325    0.5199
   m200      0.9226  1.0000  0.8507  0.8066  0.7750  0.6954  0.6965    0.5962
   m400      0.8411  0.8507  1.0000  0.8702  0.8353  0.7786  0.7872    0.7050
   m800      0.7560  0.8066  0.8702  1.0000  0.9180  0.8636  0.8690    0.8065
   m1500     0.7002  0.7750  0.8353  0.9180  1.0000  0.9281  0.9347    0.8655
   m5000     0.6195  0.6954  0.7786  0.8636  0.9281  1.0000  0.9746    0.9322
   m10000    0.6325  0.6965  0.7872  0.8690  0.9347  0.9746  1.0000    0.9432
   marathon  0.5199  0.5962  0.7050  0.8065  0.8655  0.9322  0.9432    1.0000

\end{verbatim}
}

All variables positively correlated, but less so as gap between running distances increases.
  
\end{slide}

\begin{slide}{The eigenvalues}

{\scriptsize
\begin{verbatim}


                      Eigenvalues of the Correlation Matrix

                  Eigenvalue    Difference    Proportion    Cumulative

             1    6.62214613    5.74452784        0.8278        0.8278
             2    0.87761829    0.71829715        0.1097        0.9375
             3    0.15932114    0.03527176        0.0199        0.9574
             4    0.12404939    0.04416911        0.0155        0.9729
             5    0.07988027    0.01191512        0.0100        0.9829
             6    0.06796515    0.02154562        0.0085        0.9914
             7    0.04641953    0.02381943        0.0058        0.9972
             8    0.02260010                      0.0028        1.0000

\end{verbatim}
}

Only 1st is bigger than 1, but 2nd is much bigger than others, so include that as well.
  
\end{slide}

\begin{slide}{The eigenvectors}

{\scriptsize
\begin{verbatim}
                                  Eigenvectors

                         Prin1         Prin2         Prin3         Prin4

        m100          0.317556      0.566878      0.332262      0.127628
        m200          0.336979      0.461626      0.360657      -.259116
        m400          0.355645      0.248273      -.560467      0.652341
        m800          0.368684      0.012430      -.532482      -.479999
        m1500         0.372810      -.139797      -.153443      -.404510
        m5000         0.364374      -.312030      0.189764      0.029588
        m10000        0.366773      -.306860      0.181752      0.080069
        marathon      0.341926      -.438963      0.263209      0.299512

\end{verbatim}
}

\begin{itemize}
\item \verb-Prin1- basically average of record times: ``how good a country is at running''.
\item \verb-Prin2- contrasts sprinting with distance running.
\item \verb-Prin3- contrasts longer sprints with everything else.
\item More, but not going to keep \verb-Prin3- and beyond.
\end{itemize}
  
\end{slide}

\begin{slide}{Principal component scores (selected)}

\verb-Prin1- measures ``good overall'' (negative best), \verb-Prin2- measures ``sprinting vs.\ distance'' (negative: sprinting, positive: distance). (SAS has turned this around: check back with original data.) 
{\tiny
\begin{verbatim}
   Obs    country       Prin1      Prin2
     1      ar         0.2619    -0.34488
     2      au        -2.4464    -0.21617
     4      be        -2.0413     0.26195
     8      ca        -1.7464    -0.50035
    12      ck        10.5556     1.50877
    18      fr        -2.1719    -0.50289
    19      dee       -2.5901    -0.31067
    20      dew       -2.5527    -0.41137
    29      it        -2.7269    -0.98986
    30      jp        -1.2379     0.41357
    31      ke        -2.1683     0.53371
    37      mx        -0.6785     0.84175
    43      pl        -2.0006    -0.46260
    44      pt        -0.9164     1.30473
    45      rm        -1.1965     0.53077
    53      us        -3.4306    -1.11019
    54      ru        -2.6269    -0.75696
    55      ws         7.2312    -1.90208

\end{verbatim}
}
  
\end{slide}

\begin{slide}{Component scores plot}

{\scriptsize
\begin{verbatim}
          Plot of Prin2*Prin1$country.  Symbol used is '*'.

     2 +
       |               pt krn * * cr                       * ck
       |            nz *mx * tr    * gu
       |           en* **irin
       |        ke *f**ar*c* il         * mt
       |        bec*s***d* * cn * bu
     0 +       dew cz**jp kr * tw      * png
       |     dee*p* *uhu * * ar  * ph
       |       uk **** br*cl lu    * id
 Prin2 |    us * * fruca gr
       |         it            my  th
       |                 bm *  *   ** sg
    -2 +                                         * ws
       |                       * dr
       |
       ---+-------+-------+-------+-------+-------+-------+-------+--
        -5.0    -2.5     0.0     2.5     5.0     7.5    10.0    12.5
                                    Prin1
\end{verbatim}
}
  
\end{slide}

\begin{slide}{Data TYPE=CORR}

  \begin{itemize}
  \item Just as data TYPE=DISTANCE (we used for clustering), also TYPE=CORR, used for matrices of correlations.
  \item Procedure PROC CORR will produce this as output.
  \item Example small data set (three variables):
\begin{verbatim}
 3  7 20
 4 10 16
 6 15 11
 9 18  8
\end{verbatim}
    $x_2$ is just over twice $x_1$, while $x_3$ goes down as $x_1$ and $x_2$ goes up. So expect some high positive and negative correlations.
  \end{itemize}
  
\end{slide}

\begin{slide}{Using PROC CORR}

  \begin{itemize}
  \item Code like this:

\begin{verbatim}
data xc;
  infile "xcorr.dat";
  input x1 x2 x3;

proc corr out=fred;

proc print;
\end{verbatim}
  \item PROC CORR itself produces some output (ignored) and output data set looks like this:

{\scriptsize
\begin{verbatim}
     Obs    _TYPE_    _NAME_       x1             x2          x3

      1      MEAN                5.50000     12.5000     13.7500
      2      STD                 2.64575      4.9329      5.3151
      3      N                   4.00000      4.0000      4.0000
      4      CORR       x1       1.00000      0.9705     -0.9600
      5      CORR       x2       0.97054      1.0000     -0.9980
      6      CORR       x3      -0.96001     -0.9980      1.0000
\end{verbatim}
}
\item Last 3 lines are matrix of correlations; values as expected.
  \end{itemize}
  
\end{slide}

\begin{slide}{TYPE=CORR data set}

  \begin{itemize}
  \item Full data set includes mean and SD of each variable, and number of observations for each (same for each variable). Entry in new variable \verb-_TYPE_- says what each row of numbers is.
  \item Each correlation is of {\em two} variables, so entry in row of \verb-_NAME_- and variable column says which two variables involved.
  \item Can create TYPE=CORR data set yourself. Not everything has to be specified:
    \begin{itemize}
    \item if \verb-_TYPE_- missing, CORR assumed.
    \item if \verb-_NAME_- missing, variables may not get names (but OK if planning to use all in analysis).
    \item Correlation eg.\ between $x_1$ and $x_2$ same as between $x_2$ and $x_1$, so can give redundant correlations as \verb-.- (missing).
    \item PROC PRINCOMP only uses correlations (not mean, SD or sample size --- no testing). So can still do if you have only correlations.
    \end{itemize}
  \end{itemize}
  
\end{slide}

\begin{slide}{Doing principal components with (above) correlations only}

  \begin{itemize}
  \item Create data file like this:

\begin{verbatim}
1 0.9705 -0.9600
. 1      -0.9980
. .       1
\end{verbatim}

  \item Code like below. Note: no actual data implies no component scores (and no output data set).

\begin{verbatim}
data yc(type=corr);
  infile "ycorr.dat";
  input x1 x2 x3;

proc princomp;

\end{verbatim}

  \item Can also use PROC PRINT to check proper reading of data.

  \item PROC PRINCOMP handles this kind of data automatically.
  \end{itemize}
  
\end{slide}

\begin{slide}{Output}

{\scriptsize
\begin{verbatim}
                        The PRINCOMP Procedure

                 Eigenvalues of the Correlation Matrix
 
             Eigenvalue    Difference    Proportion    Cumulative
        1    2.95242147    2.90600335        0.9841        0.9841
        2    0.04641812    0.04525771        0.0155        0.9996
        3    0.00116041                      0.0004        1.0000

                             Eigenvectors
 
                        Prin1         Prin2         Prin3
             x1      0.573007      0.811856      -.112035
             x2      0.580528      -.305586      0.754721
             x3      -.578489      0.497500      0.646408

\end{verbatim}
}

\begin{itemize}
\item Data behind correlations effectively one-dimensional.
\item Principal component made of first two variables minus third almost entirely summarizes data.
\end{itemize}
  
\end{slide}

\end{document}
