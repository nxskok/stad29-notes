\PassOptionsToPackage{unicode=true}{hyperref} % options for packages loaded elsewhere
\PassOptionsToPackage{hyphens}{url}
%
\documentclass[ignorenonframetext,]{beamer}
\usepackage{pgfpages}
\setbeamertemplate{caption}[numbered]
\setbeamertemplate{caption label separator}{: }
\setbeamercolor{caption name}{fg=normal text.fg}
\beamertemplatenavigationsymbolsempty
% Prevent slide breaks in the middle of a paragraph:
\widowpenalties 1 10000
\raggedbottom
\setbeamertemplate{part page}{
\centering
\begin{beamercolorbox}[sep=16pt,center]{part title}
  \usebeamerfont{part title}\insertpart\par
\end{beamercolorbox}
}
\setbeamertemplate{section page}{
\centering
\begin{beamercolorbox}[sep=12pt,center]{part title}
  \usebeamerfont{section title}\insertsection\par
\end{beamercolorbox}
}
\setbeamertemplate{subsection page}{
\centering
\begin{beamercolorbox}[sep=8pt,center]{part title}
  \usebeamerfont{subsection title}\insertsubsection\par
\end{beamercolorbox}
}
\AtBeginPart{
  \frame{\partpage}
}
\AtBeginSection{
  \ifbibliography
  \else
    \frame{\sectionpage}
  \fi
}
\AtBeginSubsection{
  \frame{\subsectionpage}
}
\usepackage{lmodern}
\usepackage{amssymb,amsmath}
\usepackage{ifxetex,ifluatex}
\usepackage{fixltx2e} % provides \textsubscript
\ifnum 0\ifxetex 1\fi\ifluatex 1\fi=0 % if pdftex
  \usepackage[T1]{fontenc}
  \usepackage[utf8]{inputenc}
  \usepackage{textcomp} % provides euro and other symbols
\else % if luatex or xelatex
  \usepackage{unicode-math}
  \defaultfontfeatures{Ligatures=TeX,Scale=MatchLowercase}
\fi
\usetheme[]{AnnArbor}
\usecolortheme{dove}
% use upquote if available, for straight quotes in verbatim environments
\IfFileExists{upquote.sty}{\usepackage{upquote}}{}
% use microtype if available
\IfFileExists{microtype.sty}{%
\usepackage[]{microtype}
\UseMicrotypeSet[protrusion]{basicmath} % disable protrusion for tt fonts
}{}
\IfFileExists{parskip.sty}{%
\usepackage{parskip}
}{% else
\setlength{\parindent}{0pt}
\setlength{\parskip}{6pt plus 2pt minus 1pt}
}
\usepackage{hyperref}
\hypersetup{
            pdftitle={STAD29: Statistics for the Life and Social Sciences},
            pdfauthor={Lecture notes},
            pdfborder={0 0 0},
            breaklinks=true}
\urlstyle{same}  % don't use monospace font for urls
\newif\ifbibliography
\setlength{\emergencystretch}{3em}  % prevent overfull lines
\providecommand{\tightlist}{%
  \setlength{\itemsep}{0pt}\setlength{\parskip}{0pt}}
\setcounter{secnumdepth}{0}

% set default figure placement to htbp
\makeatletter
\def\fps@figure{htbp}
\makeatother

\usepackage{multicol}

\title{STAD29: Statistics for the Life and Social Sciences}
\author{Lecture notes}
\date{}

\begin{document}
\frame{\titlepage}

\begin{frame}[fragile]

\begin{verbatim}
## Installing package into '/home/ken/R/x86_64-pc-linux-gnu-library/3.6'
## (as 'lib' is unspecified)
\end{verbatim}

\begin{verbatim}
## Warning in install.packages("mgcv", type =
## "source"): installation of package 'mgcv' had non-
## zero exit status
\end{verbatim}

\end{frame}

\hypertarget{course-outline}{%
\section{Course Outline}\label{course-outline}}

\begin{frame}[fragile]{Course and instructor}
\protect\hypertarget{course-and-instructor}{}

\begin{itemize}
\item
  Lecture: Wednesday 14:00-16:00 in HW 215. Optional computer lab Monday
  16:00-17:00 in BV 498.
\item
  Instructor: Ken Butler
\item
  Office: IC 471.
\item
  E-mail: \texttt{butler@utsc.utoronto.ca}
\item
  Office hours: Wednesday 11:00-12:00. Or make an appointment. E-mail
  always good.
\item
  Course website: \href{http://ritsokiguess.site/STAD29/}{link}.
\item
  Using Quercus for assignments/grades only; using website for
  everything else.
\end{itemize}

\end{frame}

\begin{frame}{Texts}
\protect\hypertarget{texts}{}

\begin{itemize}
\item
  There is no official text for this course.
\item
  You may find ``R for Data Science'',
  \href{http://r4ds.had.co.nz/}{\textbf{link}} helpful for R background.
\item
  I will refer frequently to my book of Problems and Solutions in
  Applied Statistics (PASIAS),
  \href{http://ritsokiguess.site/pasias/}{\textbf{link}}.
\item
  Both of these resources are and will remain free.
\end{itemize}

\end{frame}

\begin{frame}{Programs, prerequisites and exclusions}
\protect\hypertarget{programs-prerequisites-and-exclusions}{}

\begin{itemize}
\item
  Prerequisites:
\item
  For undergrads: STAC32. Not negotiable.
\item
  For grad students, a first course in statistics, and some training in
  regression and ANOVA. The less you know, the more you'll have to catch
  up!
\item
  This course is a required part of Applied Statistics minor.
\item
  Exclusions: \textbf{this course is not for Math/Statistics/CS
  majors/minors}. It is for students in other fields who wish to learn
  some more advanced statistical methods. The exclusions in the Calendar
  reflect this.
\item
  If you are in one of those programs, you won't get program credit for
  this course, \textbf{or for any future STA courses you take.}
\end{itemize}

\end{frame}

\begin{frame}{Computing}
\protect\hypertarget{computing}{}

\begin{itemize}
\item
  Computing: big part of the course, \textbf{not} optional. You will
  need to demonstrate that you can use R to analyze data, and can
  critically interpret the output.
\item
  For grad students who have not come through STAC32, I am happy to
  offer extra help to get you up to speed.
\end{itemize}

\end{frame}

\begin{frame}{Assessment 1/2}
\protect\hypertarget{assessment-12}{}

\begin{itemize}
\item
  Grading: (2 hour) midterm, (3 hour) final exam. Assignments most
  weeks, due Tuesday at 11:59pm. Graduate students (STA 1007) also
  required to complete a project using one or more of the techniques
  learned in class, on a dataset from their field of study. Projects due
  on the last day of classes.
\item
  Assessment:

  \begin{tabular}{lcc}
  & STAD29 & STA 1007\\
  Assignments & 20\% & 20\%\\
  Midterm exam & 30\%  & 20\% \\
  Project & - & 20\%\\
  Final exam & 50\% & 40\%
  \end{tabular}
\end{itemize}

\end{frame}

\begin{frame}{Assessment 2/2}
\protect\hypertarget{assessment-22}{}

\begin{itemize}
\item
  Assessments missed \emph{with documentation} will cause a re-weighting
  of other assessments of same type. No make-ups.
\item
  You \textbf{must pass the final exam} to guarantee passing the course.
  If you fail the final exam but would otherwise have passed the course,
  you receive a grade of 45.
\end{itemize}

\end{frame}

\begin{frame}{Plagiarism}
\protect\hypertarget{plagiarism}{}

\begin{itemize}
\item
  \href{http://www.utoronto.ca/academicintegrity/academicoffenses.html}{\textbf{This
  link}} defines academic offences at this university. Read it. You are
  bound by it.
\item
  Plagiarism defined (at the end) as
\end{itemize}

\begin{quote}
The wrongful appropriation and purloining, and publication as one's own,
of the ideas, or the expression of the ideas \ldots{} of another.
\end{quote}

\begin{itemize}
\item
  The code and explanations that you write and hand in must be
  \emph{yours and yours alone}.
\item
  When you hand in work, it is implied that it is \emph{your} work.
  Handing in work, with your name on it, that was actually done by
  someone else is an \emph{academic offence}.
\item
  If I am suspicious that anyone's work is plagiarized, I will take
  action.
\end{itemize}

\end{frame}

\begin{frame}[fragile]{Getting help}
\protect\hypertarget{getting-help}{}

\begin{itemize}
\item
  The English Language Development Centre supports all students in
  developing better Academic English and critical thinking skills needed
  in academic communication. Make use of the personalized support in
  academic writing skills development. Details and sign-up information:
  \href{http://www.utsc.utoronto.ca/eld/}{link}.
\item
  Students with diverse learning styles and needs are welcome in this
  course. In particular, if you have a disability/health consideration
  that may require accommodations, please feel free to approach the
  AccessAbility Services Office as soon as possible. I will work with
  you and AccessAbility Services to ensure you can achieve your learning
  goals in this course. Enquiries are confidential. The UTSC
  AccessAbility Services staff are available by appointment to assess
  specific needs, provide referrals and arrange appropriate
  accommodations: (416) 287-7560 or by e-mail:
  \texttt{ability@utsc.utoronto.ca}.
\end{itemize}

\end{frame}

\begin{frame}{Course material}
\protect\hypertarget{course-material}{}

\begin{itemize}
\tightlist
\item
  Dates and times
\item
  Regression-like things

  \begin{itemize}
  \tightlist
  \item
    review of (multiple) regression
  \item
    logistic regression (including multi-category responses)
  \item
    survival analysis
  \end{itemize}
\item
  ANOVA-like things

  \begin{itemize}
  \tightlist
  \item
    more ANOVA
  \item
    multivariate ANOVA
  \item
    repeated measures
  \end{itemize}
\item
  Multivariate methods

  \begin{itemize}
  \tightlist
  \item
    discriminant analysis
  \item
    cluster analysis
  \item
    (multidimensional scaling)
  \item
    principal components
  \item
    factor analysis
  \end{itemize}
\item
  Miscellanea

  \begin{itemize}
  \tightlist
  \item
    (time series), multiway frequency tables
  \end{itemize}
\end{itemize}

\end{frame}

\end{document}
