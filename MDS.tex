\documentclass[pdf]{prosper}
\usepackage[Lakar]{HA-prosper}
\usepackage{graphicx}


\begin{document}

\begin{slide}{Multidimensional Scaling}

  \begin{itemize}
  \item Have distances between individuals.
  \item Want to draw a picture (map) in 2 dimensions showing individuals so that distances (or order of distances) as close together as possible.
  \item If want to preserve actual distances, called {\em metric multidimensional scaling} (in SAS, \verb-level=absolute-)
  \item If only want to preserve order of distances, called {\em non-metric multidimensional scaling} (in SAS, \verb-level=ordinal-).
  \item Metric scaling has solution that can be worked out exactly.
  \item Non-metric only has iterative solution.
  \item Assess quality of fit via quantity ``stress'', whether use of resulting map is reasonable. (Try something obviously 3-dimensional and assess its failure.) Stress has min 0 and max 1.
  \end{itemize}

\end{slide}


\begin{slide}{Metric scaling: European cities}

The file \verb-europe.dat- contains road distances (in km) between 16 European cities. Can we reproduce a map of Europe from these distances?

\vspace{3ex}

First, reading in the data (as TYPE=DISTANCE):

\begin{verbatim}
data euro(type=distance);
  infile "europe.dat" delimiter=",";
  input city $ Amsterdam Athens Barcelona Berlin 
    Cologne Copenhagen Edinburgh Geneva London 
    Madrid Marseille Munich Paris Prague Rome Vienna;
\end{verbatim}

\begin{itemize}
\item Values in spreadsheet.
\item Save as .csv.
\item Take out quotes.
\item Values separated by commas, suitable for reading by SAS.
\end{itemize}

\end{slide}

\begin{slide}{The code, using PROC MDS}

\begin{verbatim}
proc mds level=absolute out=y outres=z;
proc print data=y;
proc sort data=z;
  by residual;
proc print data=z;
  var _row_ _col_ residual;
proc plot data=y;
  plot dim1 * dim2 $ _label_;
\end{verbatim}

  \begin{itemize}
  \item Run PROC MDS using \verb-level=absolute- to reproduce the exact distances (to scale).
  \item Two output data sets: one containing the coordinates for our map, and one containing the observed and predicted (from map) distances and residuals.
  \item Print coordinates.
  \item Sort residuals and print them (with the cities they belong to).
  \item Plot coordinates, labelling each point by its city.
  \end{itemize}
  
\end{slide}

\begin{slide}{The coordinates}

In Dim1 and Dim2:
{\scriptsize
\begin{verbatim}
             _TYPE_      _LABEL_      _NAME_           Dim1       Dim2
 
 1   2   .   CRITERION                                 0.07        .  
 2   2   .   CONFIG      Amsterdam    Amsterdam     -300.71     558.62
 3   2   .   CONFIG      Athens       Athens        2599.74    -375.74
 4   2   .   CONFIG      Barcelona    Barcelona     -704.34   -1012.29
 5   2   .   CONFIG      Berlin       Berlin         402.29     619.72
 6   2   .   CONFIG      Cologne      Cologne        -83.70     396.98
 7   2   .   CONFIG      Copenhagen   Copenhagen      97.17    1241.96
 8   2   .   CONFIG      Edinburgh    Edinburgh    -1232.60     906.77
 9   2   .   CONFIG      Geneva       Geneva        -185.99    -342.22
10   2   .   CONFIG      London       London        -574.43     406.08
11   2   .   CONFIG      Madrid       Madrid       -1341.37   -1088.16
12   2   .   CONFIG      Marseille    Marseille     -319.76    -750.10
13   2   .   CONFIG      Munich       Munich         326.13     -25.17
14   2   .   CONFIG      Paris        Paris         -525.60      49.92
15   2   .   CONFIG      Prague       Prague         541.20     285.90
16   2   .   CONFIG      Rome         Rome           541.38   -1031.08
17   2   .   CONFIG      Vienna       Vienna         760.58     158.80
\end{verbatim}
}
Stress 0.07 is small.
  
\end{slide}

\begin{slide}{The sorted residuals (edited)}

{\scriptsize
\begin{verbatim}
             Obs    _ROW_         _COL_         RESIDUAL

               1    Vienna        London        -445.723
               2    Edinburgh     Athens        -273.247
               3    Cologne       Athens        -230.477
               4    London        Edinburgh     -170.966
               5    Madrid        Cologne       -170.119
               6    London        Athens        -170.038
...
             115    Rome          Madrid         215.393
             116    Rome          Barcelona      225.139
             117    Madrid        Edinburgh      374.108
             118    Rome          Athens         390.827
             119    Copenhagen    Athens         434.100
             120    Edinburgh     Copenhagen     492.631
\end{verbatim}
}
Edinburgh and Athens feature in a lot of the large residuals.
  
\end{slide}

\begin{slide}{The map}

{\scriptsize
\begin{verbatim}
        Plot of Dim1*Dim2$_LABEL_.  Symbol points to label.

   4000 +
        |
        |
        |
D       |                      > Athens
i       |
m  2000 +
e       |
n       |
s       |
i       |         > Rome           Vienna <  > Prague
o       |                             > Munich     > Berlin Copenhagen
n     0 +                                      > Cologne        ^
        |     Marseille <       > Geneva          > Amsterdam
1       |          > Barcelona    Paris <      > London
        |
        |        > Madrid                                > Edinburgh
        |
  -2000 +
        -+---------+---------+---------+---------+---------+---------+
       -1500     -1000     -500        0        500      1000     1500

                                  Dimension 2

\end{verbatim}
}

  
\end{slide}

\begin{slide}{Comments on map}

  \begin{itemize}
  \item The map looks upside down!
  \item MDS doesn't know about directions, only distances, so map could come out reflected (vertically or horizontally) or rotated.
  \item Given all that, cities look in about right relative places.
  \item City pairs with largest positive residuals have large bodies of water between them (affecting road distance considerably):
    \begin{itemize}
    \item Edinburgh--Copenhagen (North Sea)
    \item Rome--Athens (Adriatic) 
    \end{itemize}
  \item As it happens, plotting \verb-Dim2*Dim1- produces almost reasonable map:
  \end{itemize}
  
\end{slide}

\begin{slide}{Map 2}
  {\scriptsize
\begin{verbatim}
         Plot of Dim2*Dim1$_LABEL_.  Symbol points to label.

        |
   2000 +
        |
        |
D       |                         > Copenhagen
i  1000 +         > Edinburgh
m       |                   Cologne Berlin
e       |          London <  v  ^     ^Prague
n       |                Amsterdam     ^  > Vienna
s     0 +                  > Paris   > Munich
i       |                      > Geneva
o       |                                                Athens <
n       |                    > Marseille
  -1000 + Madrid <       > Barcelona   > Rome
2       |
        |
        |
  -2000 +
        |
        -+-----------+-----------+-----------+-----------+-----------+
       -2000       -1000         0         1000        2000       3000

                                  Dimension 1

\end{verbatim}
}
\end{slide}


\begin{slide}{Non-metric scaling: languages}

  \begin{itemize}
  \item Recall language data (from cluster analysis): 1--10, measure dissimilarity between two languages by how many number names {\em differ} in first letter. Data:

\begin{verbatim}
en 0 2 2  7 6  6   6  6  7  9 9
no 2 0 1  5 4  6   6  6  7  8 9
dk 2 1 0  6 5  6   5  5  6  8 9
nl 7 5 6  0 5  9   9  9 10  8 9
de 6 4 5  5 0  7   7  7  8  9 9
fr 6 6 6  9 7  0   2  1  5 10 9
es 6 6 5  9 7  2   0  1  3 10 9
it 6 6 5  9 7  1   1  0  4 10 8
pl 7 7 6 10 8  5   3  4  0 10 9
hu 9 8 8  8 9  10 10 10 10  0 8
sf 9 9 9  9 9  9   9  8  9  8 0

\end{verbatim}

\item Only want to reproduce {\em order} of dissimilarities; actual numbers don't matter. (Map only reproduces {\em relative} closeness of languages.)

  \end{itemize}
  
\end{slide}

\begin{slide}{Code}

  \begin{itemize}
  \item Read data as distances, use \verb-level=ordinal-. Print coordinates and residuals, plot map (labelled by language):

\begin{verbatim}
data lang(type=distance);
  infile "one-ten.dat";
  input lang $ en no dk nl de fr es it pl hu sf;

proc mds level=ordinal out=coords outres=dist;
  id lang;

proc print data=dist;
  var _row_ _col_ data distance residual;

proc print data=coords;

proc plot data=coords;
  plot dim2 * dim1 = '*' $ lang;

\end{verbatim}
  \end{itemize}
  
\end{slide}

\begin{slide}{Output from PROC MDS}

{\scriptsize
\begin{verbatim}
         Multidimensional Scaling:  Data=WORK.LANG.DATA
          Shape=TRIANGLE Condition=MATRIX Level=ORDINAL
            Coef=IDENTITY Dimension=2 Formula=1 Fit=1
  Mconverge=0.01 Gconverge=0.01 Maxiter=100 Over=2 Ridge=0.0001
                       Badness-             Convergence Measures
                         of-Fit  Change in  --------------------
 Iteration  Type      Criterion  Criterion   Monotone   Gradient
       0    Initial      0.2009          .          .          .
       1    Monotone     0.1478     0.0531     0.1358     0.6781
       2    Gau-New      0.1126     0.0352          .          .
       3    Monotone     0.1020     0.0105     0.0483     0.3363
       4    Gau-New      0.0997   0.002376          .          .
       5    Monotone     0.0928   0.006869     0.0374     0.2226
       6    Gau-New      0.0923   0.000483          .          .
       7    Monotone     0.0915   0.000823     0.0138     0.2190
       8    Gau-New      0.0914  0.0000983          .          .
       9    Monotone     0.0910   0.000349   0.009497     0.2341
      10    Gau-New      0.0888   0.002191          .     0.0533
      11    Gau-New      0.0887   0.000106          .     0.0169
      12    Gau-New      0.0887  0.0000126          .   0.006850
\end{verbatim}
}
Iterative procedure converges (stress stops getting smaller at 0.0887, which is small). 

  
\end{slide}

\begin{slide}{The residuals (selected)}

Shown: pair of languages, dissimilarity, distance on map, residual (based on ordered data). Large residual means data and distance on map don't match.
{\scriptsize
\begin{verbatim}
      Obs    _ROW_    _COL_    DATA    DISTANCE    RESIDUAL

        7     de       en        6      0.81928     0.49528
       55     sf       hu        8      2.00452     0.35904
       49     sf       nl        9      3.15422    -0.34249
       40     hu       nl        8      2.02361     0.33995
       48     sf       dk        9      2.48611     0.32562
       31     pl       dk        6      1.62422    -0.30966
        6     nl       dk        6      1.61869    -0.30413
       50     sf       de        9      3.10815    -0.29643
        5     nl       no        5      1.31502    -0.27280
       32     pl       nl       10      3.23354     0.24178
       54     sf       pl        9      2.54350     0.26823
\end{verbatim}
}

\begin{itemize}
\item Positive residual: actual dissimilarity greater than expected (compared to map)
\item Negative residual: actual dissimilarity less than expected from map.
\end{itemize}
  
\end{slide}

\begin{slide}{The coordinates}

{\scriptsize
\begin{verbatim}
  Obs _DIMENS_ _MATRIX_ _TYPE_    lang _NAME_   Dim1     Dim2

    1     2        .    CRITERION              0.08872   .
    2     2        .    CONFIG     en    en    0.30099  0.65225
    3     2        .    CONFIG     no    no   -0.11417  0.58068
    4     2        .    CONFIG     dk    dk    0.08220  0.30450
    5     2        .    CONFIG     nl    nl   -1.30472  1.13912
    6     2        .    CONFIG     de    de   -0.39587  1.08307
    7     2        .    CONFIG     fr    fr    1.22529  0.07596
    8     2        .    CONFIG     es    es    1.12900 -0.15541
    9     2        .    CONFIG     it    it    0.96244 -0.35587
   10     2        .    CONFIG     pl    pl    1.33098 -0.73409
   11     2        .    CONFIG     hu    hu   -2.33345 -0.60349
   12     2        .    CONFIG     sf    sf   -0.88268 -1.98673

\end{verbatim}
}

\begin{itemize}
\item 1st row: stress value (max 1, min 0).
\item CONFIG lines: Dim1 and Dim2 have coordinates.
\end{itemize}
  
\end{slide}

\begin{slide}{The map}

{\scriptsize
\begin{verbatim}
           Plot of Dim2*Dim1$lang.  Symbol points to label.

     |
D  2 +
i    |
m    |
e    |                     > nl       > de
n    |                                    > no > en
s    |                                      > dk
i  0 +                                                es <> fr
o    |                                                 > it
n    |         > hu                                        > pl
     |
2    |
     |
  -2 +                          > sf
     --+-----------+-----------+-----------+-----------+-----------+--
      -3          -2          -1           0           1           2

                                Dimension 1
\end{verbatim}
}
  
\end{slide}

\begin{slide}{Comments on map}

  \begin{itemize}
  \item See how distant Hungarian and Finnish are from each other, and the rest.
  \item See tight grouping of Italian, French and Spanish (Polish nearby).
  \item See looser grouping of Germanic languages at top (English, German, Dutch, Norwegian, Danish). 
  \end{itemize}
  
\end{slide}

\begin{slide}{Guidelines for stress values}

Smaller is better:

\begin{tabular}{lp{3in}}
  Stress value & Interpretation \\
  \hline
  Less than 0.05 & Excellent: no prospect of misinterpretation (rarely achieved)\\
  0.05--0.10 & Good: most distances reproduced well, small prospect of false inferences\\
0.10--0.20 & Fair: usable, but some distances misleading.\\
More than 0.20 & Poor: may be dangerous to interpret\\
\hline
\end{tabular}

\begin{itemize}
\item Cities and languages examples both had stress in ``good'' range.
\end{itemize}
  
\end{slide}

\begin{slide}{A cube}

\begin{verbatim}
  a-----b
  |\    |\
  | c---- d
  | |   | |
  e-|---f |
   \|    \|
    g-----h
\end{verbatim}

Cube has side length 1, so distance across diagonal on same face is $\sqrt{2}\simeq 1.4$ and ``long'' diagonal of cube is $\sqrt{3}\simeq 1.7$. 
  
\vspace{3ex}

Try MDS on this obviously 3-dimensional data.

\end{slide}

\begin{slide}{Converges OK}

{\scriptsize
\begin{verbatim}
                            Badness-
                                of-Fit    Change in    Convergence
    Iteration    Type        Criterion    Criterion        Measure
    --------------------------------------------------------------
          0      Initial        0.2987            .         0.6106
          1      Lev-Mar        0.2275       0.0711         0.1308
          2      Gau-New        0.2251     0.002446         0.0409
          3      Gau-New        0.2248     0.000263         0.0164
          4      Gau-New        0.2248    0.0000426       0.006667
\end{verbatim}
}

but stress, at 0.2248, in ``poor'' range. Map probably won't reproduce cube very well.
  
\end{slide}

\begin{slide}{``Map'' of cube}

{\scriptsize
\begin{verbatim}
         Plot of Dim2*Dim1$_LABEL_.  Symbol points to label.

          |
     D  1 +
     i    |
     m    |                    > b         > c
     e    |       > d                                   > a
     n    |
     s    |
     i  0 +
     o    |
     n    |       > h                                   > e
          |
     2    |                  > f             > g
          |
       -1 +
          ---+-----------+-----------+-----------+-----------+--
           -1.0        -0.5         0.0         0.5         1.0

                                Dimension 1

\end{verbatim}
}


  
\end{slide}

\begin{slide}{Comments}

  \begin{itemize}
  \item Map doesn't resemble cube.
  \item Some of the residuals are large: eg.\ \verb-g- and \verb-f-: actual distance is 1.4, map distance 0.7.
  \item Might have guessed this with stress in ``poor'' range.
  \item SAS lets you choose dimension of map. Use this PROC MDS line:
 
\begin{verbatim}
proc mds dim=3 level=absolute outres=res2;
\end{verbatim}

(no point saving coordinates since we cannot plot them.)
\item Resulting stress is 0.0342, ``excellent''.
\item Largest residual (in size) is $-0.1$, most much smaller.
\item Can't ``squeeze'' 3-D data into 2 dimensions!
  \end{itemize}
  
\end{slide}


\end{document}
