\documentclass[pdf]{prosper}

\usepackage[lakar]{HA-prosper}

\usepackage{graphicx}

% compile with latex and then dvips -t letter

%\setlength{\parindent}{0em}
%\setlength{\parskip}{30pt}

%\setlength{\textwidth}{5in}
%\setlength{\textheight}{5in}

%\setcounter{secnumdepth}{0}


\begin{document}

\begin{slide}{Survival analysis}

  \begin{itemize}
  \item So far, have seen:
    \begin{itemize}
    \item response variable counted or measured (regression)
    \item response variable categorized (logistic regression)
    \end{itemize}
    and have predicted response from explanatory variables.
  \item But what if response is time until event (eg.\ time of
    survival after surgery)?
  \item Additional complication: event might not have happened at end of study (eg.\ patient still alive). But knowing that patient has ``not died yet'' presumably informative. Such data called {\em censored}. 
  \item Enter {\em survival analysis}, in particular the ``Cox proportional hazards model''. 
  \item Explanatory variables in this context often called {\em covariates}.
  \end{itemize}

\end{slide}

\begin{slide}{Example: still dancing?}

  \begin{itemize}
  \item 12 women who have just started taking dancing lessons are
    followed for up to a year, to see whether they are still taking
    dancing lessons (or have quit).
  \item This might depend on:
    \begin{itemize}
    \item a treatment (visit to a dance competition)
    \item woman's age (at start of study).
    \end{itemize}
  \item Data:

{\scriptsize
\begin{verbatim}
Months Dancing Treatment Age
   1      1        0      16
   2      1        0      24
   2      1        0      18
   3      0        0      27
   4      1        0      25
   5      1        0      21
  11      1        0      55
   7      1        1      26
   8      1        1      36
  10      1        1      38
  10      0        1      45
  12      1        1      47
\end{verbatim}
}

  \end{itemize}
  
\end{slide}

\begin{slide}{About the data}

  \begin{itemize}
  \item \verb-months- and \verb-dancing- are kind of combined response:
    \begin{itemize}
    \item  \verb-Months- is number of months a woman was actually observed dancing
    \item \verb-dancing- is 1 if woman quit, 0 if still dancing at end of study.
    \end{itemize}
  \item Treatment is 1 if woman went to dance competition, 0 otherwise.
  \item Want to do predictions for probabilities of still dancing after 3, 6, 9, 12 months for treatment group and control group, for women of ages 25 and 45.
\end{itemize}
\end{slide}

\begin{slide}{Doing predictions}

Add to data file:
{\scriptsize
\begin{verbatim}
3 . 0 25
6 . 0 25
9 . 0 25
12 . 0 25
...
3 . 1 45
6 . 1 45
9 . 1 45
12 . 1 45
\end{verbatim}
} 


\vspace{3ex}

Gives predicted survival probabilities for 3, 6, 9 and 12 months for
(a) woman aged 25 in control group, (b) women aged 45 in traatment
group (do other age/treatment combos also).

\vspace{2ex}

Censoring variable missing for these: won't affect analysis.

\end{slide}

\begin{slide}{The code}

\begin{verbatim}
data dancers;
  infile "survival1.dat";
  input months dancing treatment age;

proc phreg;
  model months*dancing(0) = age treatment;
  output out=fred survival=s;

proc print data=fred;
\end{verbatim}

\vspace{3ex}

  \begin{itemize}
  \item 
Nothing new in reading data.
\item Note specification of model: includes both survival time and censoring variable in response, and indication of what value means ``censored''.
\item As ever, predictions saved in output data set, then printed.
  \end{itemize}
  
\end{slide}

\begin{slide}{The output, edited}

{\scriptsize
\begin{verbatim}
             Model Information

   Data Set                 WORK.DANCERS
   Dependent Variable       months
   Censoring Variable       dancing
   Censoring Value(s)       0
   Ties Handling            BRESLOW

  Number of Observations Read          28
  Number of Observations Used          12

 Summary of the Number of Event and Censored Values

                                    Percent
  Total       Event    Censored    Censored
     12          10           2       16.67
\end{verbatim}
}
  
\end{slide}

\begin{slide}{Output part 2}

{\scriptsize
\begin{verbatim}
          Testing Global Null Hypothesis: BETA=0

  Test                 Chi-Square       DF     Pr > ChiSq
  Likelihood Ratio        21.0016        2         <.0001
  Score                   14.2093        2         0.0008
  Wald                     5.5556        2         0.0622

                  Analysis of Maximum Likelihood Estimates

                     Parameter  Standard                         Hazard
 Parameter    DF      Estimate     Error  Chi-Square  Pr > ChiSq  Ratio
 age           1      -0.35284   0.14973      5.5532      0.0184  0.703
 treatment     1      -4.28283   2.54084      2.8412      0.0919  0.014

\end{verbatim}
}

\begin{itemize}
\item Overall model seems significant.
\item Survival depends on age but not apparently on treatment (could be small size of data set or confounding of treatment with age).
\end{itemize}

  
\end{slide}

\begin{slide}{Predicted survival probs}

{\scriptsize
\begin{verbatim}
Obs    months    dancing    treatment    age       s
 13       3         .           0         25    0.87856
 14       6         .           0         25    0.56647
 15       9         .           0         25    0.00000
 16      12         .           0         25    0.00000
 
 17       3         .           1         25    0.99821
 18       6         .           1         25    0.99219
 19       9         .           1         25    0.00000
 20      12         .           1         25    0.00000

 21       3         .           0         45    0.99989
 22       6         .           0         45    0.99951
 23       9         .           0         45    0.14589
 24      12         .           0         45    0.00000

 25       3         .           1         45    1.00000
 26       6         .           1         45    0.99999
 27       9         .           1         45    0.97378
 28      12         .           1         45    0.08223
\end{verbatim}
}
  
\end{slide}

\begin{slide}{Conclusions from predicted probs}

  \begin{itemize}
  \item Older women more likely to be still dancing than younger women
    (compare ``profiles'' for same treatment group).
  \item Effect of treatment seems to be to increase prob of still dancing (compare ``profiles'' for same age for treatment group vs.\ not)
  \item Would be nice to see this on a graph.
  \end{itemize}
  
\end{slide}



\begin{slide}{Another way of doing predictions}

Instead of adding lines to data file and creating an output data set, use \verb-baseline- command like this:

{\scriptsize
\begin{verbatim}
data dancers;
  infile "survival1.dat";
  input months dancing treatment age;

data mypred;
  input treatment age;
  datalines;
  0 25 
  0 45 
  1 25 
  1 45 
;

proc phreg data=dancers;
  model months*dancing(0) = age treatment;
  baseline out=fred covariates=mypred survival=s lower=lcl upper=ucl / 
nomean;
   
proc print data=fred;
\end{verbatim}
}  

\end{slide}

\begin{slide}{Results, including CIs}

{\scriptsize
\begin{verbatim}
  Obs    age    treatment    months       s         lcl        ucl
    1     25        0           0      1.00000     .          .     
    2     25        0           1      0.96633    0.90266    1.00000
    3     25        0           2      0.79225    0.60826    1.00000
    4     25        0           4      0.63726    0.35919    1.00000
    5     25        0           5      0.14748    0.05834    0.37282
    6     25        0           7      0.00000    0.00000    1.00000
    7     25        0           8      0.00000    0.00000    1.00000
    8     25        0          10      0.00000    0.00000    1.00000
    9     25        0          11      0.00000    0.00000    1.00000
   10     25        0          12      0.00000     .          .     
 
   11     45        0           0      1.00000     .          .     
   12     45        0           1      0.99997    0.99980    1.00000
   13     45        0           2      0.99980    0.99895    1.00000
   14     45        0           4      0.99961    0.99760    1.00000
   15     45        0           5      0.99835    0.99486    1.00000
   16     45        0           7      0.75954    0.52629    1.00000
   17     45        0           8      0.04468    0.00002    1.00000
   18     45        0          10      0.00001    0.00000    1.00000
   19     45        0          11      0.00000    0.00000    1.00000
   20     45        0          12      0.00000     .          .     
\end{verbatim}
}
\end{slide}

\begin{slide}{The rest}

{\scriptsize
\begin{verbatim}
   21     25        1           0      1.00000     .          .     
   22     25        1           1      0.99953    0.99727    1.00000
   23     25        1           2      0.99679    0.98545    1.00000
   24     25        1           4      0.99380    0.96712    1.00000
   25     25        1           5      0.97393    0.92908    1.00000
   26     25        1           7      0.01220    0.00080    0.18538
   27     25        1           8      0.00000    0.00000    1.00000
   28     25        1          10      0.00000    0.00000    1.00000
   29     25        1          11      0.00000    0.00000    1.00000
   30     25        1          12      0.00000     .          .     

   31     45        1           0      1.00000     .          .     
   32     45        1           1      1.00000    1.00000    1.00000
   33     45        1           2      1.00000    0.99998    1.00000
   34     45        1           4      0.99999    0.99995    1.00000
   35     45        1           5      0.99998    0.99990    1.00000
   36     45        1           7      0.99621    0.98945    1.00000
   37     45        1           8      0.95800    0.88352    1.00000
   38     45        1          10      0.84737    0.67929    1.00000
   39     45        1          11      0.38657    0.09793    1.00000
   40     45        1          12      0.00000     .          .     

\end{verbatim}
}
  
\end{slide}

\begin{slide}{Making a plot}

  \begin{itemize}
  \item Start from output data set ``fred''.
  \item Create a new data set with everything in ``fred'' plus a new variable ``label'' which will actually be plotted
  \item Plot the predicted survival probs against ``months'' marking each point using the labels.

\begin{verbatim}
data y;
  set fred;
  label=cat(age,treatment);

proc plot;
  plot s*months $ label;
\end{verbatim}
    \item Each plotted point will be labelled with something like 450 (45 year old woman in control group).
  \end{itemize}
  
\end{slide}

\begin{slide}{The plot}

{\tiny
\begin{verbatim}
                          Plot of s*months$label.  Symbol points to label.

     s |450451  251    251           451    251
   1.0 +  4 250  3 450  3 450     251 3 450  3 450         > 451
       | 251  451> 250 451                  451                   > 451
       |
       |                                                                        > 451
       |                > 250                              > 450
       |
       |
       |                              > 250
       |
   0.5 +
       |
       |                                                                               > 451
       |
       |
       |
       |                                     > 250
       |
       |                                                       251> 450        251    450  451250
   0.0 +                                               251 2 250  2 250     450 3 250  3 251  4
       ---+------+------+------+------+------+------+------+------+------+------+------+------+--
          0      1      2      3      4      5      6      7      8      9     10     11     12

                                                 months

NOTE: 3 label characters hidden.


\end{verbatim}
}
  
\end{slide}

\begin{slide}{Discussion}

  \begin{itemize}
  \item The confidence intervals are very wide (a lack of data).
  \item ``baseline'' doesn't require any specification of lifetimes, and output gives more detail.
  \item Look at combinations of treatment and age, see where predicted survival probs ``drop off'':
    \begin{itemize}
    \item age 25, control: after 4 months
    \item age 25, treatment: after 5 months
    \item age 45, control:  after 7 months
    \item age 45, treatment: after 10 months
    \end{itemize}
  \item Indicates definite effect of age, possible effect of treatment.
  \item Suggests would be worthwhile to do another experiment with more women to get more definite conclusions.
  \end{itemize}
  
\end{slide}


\end{document}
























