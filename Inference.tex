\documentclass[pdf]{prosper}

\usepackage[Lakar]{HA-prosper}

\usepackage{graphicx}

% compile with latex and then dvips -t letter

%\setlength{\parindent}{0em}
%\setlength{\parskip}{30pt}

%\setlength{\textwidth}{5in}
%\setlength{\textheight}{5in}

%\setcounter{secnumdepth}{0}


\begin{document}

\part{Statistical Inference}

\begin{slide}{The statistical world}
  \begin{itemize}
  \item 
  Consists of:

  \begin{itemize}
  \item objects or people of interest to us ({\em individuals})
  \item things measured or counted on those individuals ({\em variables})
  \end{itemize}

\vspace{3ex}

\item About the individuals:

\begin{itemize}
\item which ones do we care about? All of them, the {\em population}.
\item which ones do we know about? The ones we happened to look at, the {\em sample}.
\end{itemize}

\vspace{3ex}

\item Sample is (or should be) randomly chosen from population, with no favoritism.
  \end{itemize}

\end{slide}

\begin{slide}{Sample to population: confidence interval}
  
  \begin{itemize}
  \item 
Want to know about population (parameter), but don't. Only have sample (statistic). Eg.\ population mean, only have sample mean.
\item Logic:
  \begin{itemize}
  \item {\em If} we knew about population, could figure out kinds of samples that might appear (math).
  \item In particular, can figure how far apart sample statistic and population parameter might be.
  \item Use this to construct {\em confidence interval} for population parameter: says eg.\ ``based on my sample, I think population mean between $a$ and $b$''. 
  \end{itemize}
\end{itemize}
\end{slide}

\begin{slide}{Test of significance}
\begin{itemize}
\item Or: 
  \begin{itemize}
  \item 
might have theory leading to {\em null hypothesis} (eg.\ population mean is 20) and {\em alternative hypothesis} (eg.\ population mean not 20).
\item This leads to {\em test of significance} (hypothesis test): ``based on my sample, I think pop.\ mean is (is not) 20''
\item Done by choosing $\alpha$ (eg.\ 0.05), calculating {\em test statistic} and {\em P-value}. If P-value $< \alpha$, {\em reject null}: have evidence in favour of alternative.
  \end{itemize}
\item Math producing inference procedures can be difficult, but calculations (with software) and interpretations need not be.
  \end{itemize}


\end{slide}

\end{document}






