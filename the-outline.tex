\documentclass{article}

\usepackage{hyperref}

\begin{document}

\section*{STAD29 / STA 1007 Course Outline}

\subsection*{Course and instructor}
\begin{itemize}
    \item  Lecture: Wednesday 14:00-16:00 in IC 326 (UTSC).
    \item  Instructor: Ken Butler
    \item  Office: IC 471.
    \item  E-mail: \verb-butler@utsc.utoronto.ca-
    \item Office hours: Wednesday mornings good, or after
      class. Or Thursday afternoon. I am often around. See if I'm in. Or make an
      appointment. E-mail always good.
    \item Course website: 
\url{www.utsc.utoronto.ca/~butler/d29}.
    \item Using Blackboard for grades only; using website for
      everything else.
\end{itemize}



\subsection*{Text, programs, prerequisites and exclusions}

\begin{itemize}
    \item {\bf Recommended text} for this course is my own (yes, I
      know):
      \url{www.utsc.utoronto.ca/~butler/r/r-howto.pdf}.
      ``The book'', free for the download.

    \item Prerequisites: for undergrads STAB22 - STAB27 - STAC32
      (or STA220-STA221). For grad students,
      a first course, and some training in
      regression and ANOVA. We can be flexible about this, since there
      is some review anyway.
    \item This course is part of Applied Statistics minor.
    \item Exclusions: this course not for Statistics majors/minors. It
      is for
      students in other fields who wish to learn some more advanced
      statistical methods. The exclusions in the Calendar reflect
      this. 
\end{itemize}
  





\subsection*{Computing and assessment}
\begin{itemize}
\item Computing: big part of the course, {\bf not}
  optional. Demonstrate that you can use 
  R to analyze data, and can
  critically interpret the output. {\em No} prior knowledge of R is assumed.
\item Grading: (3 hour) final exam, but \emph{no} midterm. There will
  be assignments most weeks. 
  Graduate students (STA 1007) also required to
  complete a project using one or more of the techniques learned in
  class, on a dataset from their field of study.    Projects due at the
  last class of the semester.

\item Assessment:

  \begin{tabular}{lcc}
    & STAD29 & STA 1007\\
    Assignments & 50\% & 50\%\\
    Project & - & 20\%\\
    Final exam & 50\% & 30\%
  \end{tabular}

\end{itemize}


\subsection*{Plagiarism}

  \begin{itemize}
  \item
    \url{http://www.utoronto.ca/academicintegrity/academicoffenses.html}
    defines academic offences at this university. Read it.
  \item Plagiarism is defined (at the end) as
    \begin{quote}
       The wrongful appropriation and purloining, and publication as one’s own, of the ideas, or the expression of the ideas ... of another.
    \end{quote}
    \item 
  For help with your assignments, you may consult only the instructor,
 your textbook, and your class
 notes.  You may not consult any other source.

The code and
    explanations  that
    you write and hand in must be \emph{yours and yours
      alone}. 
    \item When you hand in work, it is implied that it is
    \emph{your} work. Handing in work, with your name on it, that was actually done by
    someone else is an \emph{academic offence}.
  \item If I am suspicious
    that anyone's work is plagiarized, I will take action.
    
  \end{itemize}
  



\end{document}



