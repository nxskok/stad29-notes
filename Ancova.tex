\documentclass[pdf]{prosper}
\usepackage[Lakar]{HA-prosper}
\usepackage{graphicx}

\begin{document}

\begin{slide}{Analysis of covariance}

  \begin{itemize}
  \item ANOVA: explanatory variables categorical (divide data into groups)
  \item traditionally, analysis of covariance has categorical $x$'s plus one numerical $x$ (``covariate'') to be adjusted for.
  \item \texttt{lm} handles this too.
  \item Simple example: two treatments (drugs) (\verb-a- and \verb-b-), with before and after scores. 
    \begin{itemize}
    \item 
Does knowing before score and/or treatment help to predict after score?
\item Is after score different by treatment/before score?
    \end{itemize}
  \end{itemize}

\end{slide}

\begin{slide}{Data}

Treatment, before, after:

{\scriptsize
\begin{verbatim}
a 5 20
a 10 23
a 12 30
a 9 25
a 23 34
a 21 40
a 14 27
a 18 38
a 6 24
a 13 31
b 7 19
b 12 26
b 27 33
b 24 35
b 18 30
b 22 31
b 26 34
b 21 28
b 14 23
b 9 22
\end{verbatim}
  }
\end{slide}

\begin{slide}[fragile]{Matched pairs}

One way to handle this is as matched pairs: take the difference after
minus before and analyze \emph{those} (predicting the differences from
\texttt{drug}:

{\small
\begin{verbatim}
> prepost=read.table("ancova.txt",header=T)
> head(prepost)
  drug before after
1    a      5    20
2    a     10    23
3    a     12    30
4    a      9    25
5    a     23    34
6    a     21    40
> attach(prepost)
> diff=after-before
> prepost.0=aov(diff~drug)
> summary(prepost.0)
            Df Sum Sq Mean Sq F value   Pr(>F)    
drug         1  180.0   180.0   22.22 0.000173 ***
Residuals   18  145.8     8.1                     
---
Signif. codes:  0 ‘***’ 0.001 ‘**’ 0.01 ‘*’ 0.05 ‘.’ 0.1 ‘ ’ 1 
\end{verbatim}
}
  
\end{slide}

\begin{slide}{SAS code}

\begin{verbatim}
options linesize=75;

data drugs;
  infile "ancova.dat";
  input drug $ before after;

proc means;
  class drug;

proc glm;
  class drug;
  model after = drug before drug*before;
\end{verbatim}

  \begin{itemize}
  \item Get means of before and after scores for each treatment.
  \item Make sure \verb-drug- treated as categorical (``class'')
  \item Before score treated as numeric by default
  \item Interaction means ``effect of before score on after score is different for each treatment''. Fit this first.
  \end{itemize}
  
\end{slide}

\begin{slide}{The means}
{\scriptsize
\begin{verbatim}
                            The MEANS Procedure

             N
drug       Obs   Variable    N           Mean        Std Dev 
-------------------------------------------------------------
a           10   before     10     13.1000000      6.0452001 
                 after      10     29.2000000      6.6131183 

b           10   before     10     18.0000000      7.1492035 
                 after      10     28.1000000      5.4660569 
-------------------------------------------------------------

\end{verbatim}
}

\begin{itemize}
\item Mean ``after'' score slightly higher for treatment A.
\item Mean ``before'' score much higher for treatment B.
\item Greater {\em improvement} on treatment A. 
\end{itemize}
  
\end{slide}

\begin{slide}{Testing for interaction}

{\scriptsize
\begin{verbatim}
                               The GLM Procedure
Dependent Variable: after
                                       Sum of
 Source              DF        Squares    Mean Square   F Value   Pr > F

 Model                3    558.5668744    186.1889581     27.09   <.0001
 Error               16    109.9831256      6.8739453
 Corrected Total     19    668.5500000

 Source              DF      Type I SS    Mean Square   F Value   Pr > F

 drug                 1      6.0500000      6.0500000      0.88   0.3621
 before               1    540.1797947    540.1797947     78.58   <.0001
 before*drug          1     12.3370798     12.3370798      1.79   0.1991


 Source              DF    Type III SS    Mean Square   F Value   Pr > F

 drug                 1      1.2105592      1.2105592      0.18   0.6803
 before               1    552.3578682    552.3578682     80.36   <.0001
 before*drug          1     12.3370798     12.3370798      1.79   0.1991


\end{verbatim}
}
  
\end{slide}

\begin{slide}{Taking out interaction}

  \begin{itemize}
  \item Take out non-significant interaction.
  \item Assuming linear dependence of after score on before score has {\em same slope} for both treatments (though possibly different intercept).
  \item Get predicted means for ``after'' score depending on \verb-drug- and \verb-before-.
  \item Also get means for treatments ``adjusted'' for before score.

  \item Code:
\begin{verbatim}
proc glm;
  class drug;
  model after = drug before;
  output out=z predict=pred;
  lsmeans drug;

proc print data=z;

\end{verbatim}
  \end{itemize}
  
\end{slide}

\begin{slide}{Results}

{\scriptsize
\begin{verbatim}
Dependent Variable: after   
                                     Sum of
 Source                    DF       Squares   Mean Square  F Value  Pr > F
 Model                      2   546.2297947   273.1148973    37.96  <.0001
 Error                     17   122.3202053     7.1953062                 
 Corrected Total           19   668.5500000                               

 Source                    DF     Type I SS   Mean Square  F Value  Pr > F

 drug                       1     6.0500000     6.0500000     0.84  0.3720
 before                     1   540.1797947   540.1797947    75.07  <.0001

 Source                    DF   Type III SS   Mean Square  F Value  Pr > F

 drug                       1   115.3059567   115.3059567    16.03  0.0009
 before                     1   540.1797947   540.1797947    75.07  <.0001

\end{verbatim}
}
  
\end{slide}

\begin{slide}{Interpreting the output}

  \begin{itemize}
  \item Requires care!
  \item Model as a whole is significant.
  \item Type I SS says ``is each variable significant when added {\em in order}: that is:
    \begin{itemize}
    \item \verb-drug- added to a model containing nothing (not sig)
    \item \verb-before- added to model containing only \verb-drug- (sig)
    \end{itemize}
  \item Not really what we want to know.
  \item Type III SS: ``can I take this variable out of a model
    containing everything?'' Answer in both cases no. Interpretation:
    once you allow for before score, there is a significant difference
    between treatments. (But if you don't allow for before score, there
    isn't.)
  \end{itemize}
  
\end{slide}

\begin{slide}{LS-means}
Sample means for each treatment close:

{\scriptsize
\begin{verbatim}
             N
drug       Obs   Variable    N           Mean        Std Dev 
-------------------------------------------------------------
a           10   after      10     29.2000000      6.6131183 
b           10   after      10     28.1000000      5.4660569 
-------------------------------------------------------------

\end{verbatim}
}

``Least squares means'': mean score for each treatment, after allowing for difference in before scores:

{\scriptsize
\begin{verbatim}
                             The GLM Procedure
                            Least Squares Means

                           drug    after LSMEAN
                           a         31.2273292
                           b         26.0726708
\end{verbatim}
}

Treatment A noticeably (significantly) better than B, {\em once you allow for before score}.
  
\end{slide}

\begin{slide}{Looking at the predictions}

Some of them, arranged in before score order:

{\scriptsize
\begin{verbatim}
             Obs    drug    before      pred
               4     a         9      25.8073
               3     a        12      28.2898
               7     a        14      29.9447
               8     a        18      33.2547
               6     a        21      35.7371

              20     b         9      20.6527
              12     b        12      23.1351
              19     b        14      24.7901
              15     b        18      28.1000
              18     b        21      30.5824

\end{verbatim}
}

\begin{itemize}
\item Prediction for treatment A about 5 units higher than for treatment B at the same before score --- same difference as between LSMEANS.
\item Consistent because no interaction.
\item If interaction had been included, A might be higher for some before scores and B higher for others: clouds interpretation.
\end{itemize}
  
\end{slide}

\end{document}
