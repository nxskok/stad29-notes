\documentclass[pdf]{prosper}
\usepackage[Lakar]{HA-prosper}
\usepackage{graphicx}

\begin{document}

\begin{slide}{Analysis of variance}

  \begin{itemize}
  \item Analysis of variance used with:
    \begin{itemize}
    \item counted/measured response
    \item categorical explanatory variable(s)
    \item that is, data divided into groups, and see if response significantly different among groups
    \item or, see whether knowing group membership helps to predict response.
    \end{itemize}
  \item Typically two stages:
    \begin{itemize}
    \item $F$-test to detect {\em any} differences among/due to groups
    \item if $F$-test significant, do {\em multiple comparisons} to see which groups significantly different from which.
    \item Need special multiple comparisons method because just doing (say) two-sample $t$-tests on each pair of groups gives too big a chance of finding ``signficant'' differences by accident.
    \end{itemize}
  \end{itemize}
  
\end{slide}

\begin{slide}{Example: jumping rats}

  \begin{itemize}
  \item Link between exercise and healthy bones: exercise stresses bones and helps them grow stronger.
  \item Study assessed effect of jumping on bone density of rats. Rats randomly assigned to one of 3 treatment groups:
    \begin{itemize}
    \item no jumping (control)
    \item low-jump (30 cm)
    \item high-jump (60 cm)
    \end{itemize}
  \item 8 jumps/day, 5 days/week, measure bone density (response) at end.
  \item PROC GLM to analyze (or PROC ANOVA, only works for balanced designs).
  \end{itemize}
  
\end{slide}

\begin{slide}{The data}

  \begin{itemize}
  \item Some of the data (10 rats in each group). Data separated by tabs.
\begin{verbatim}
Control      1            603
Control      1            569
...
Lowjump      2            635
Lowjump      2            605
...
Highjump     3            643
Highjump     3            650
\end{verbatim}
\end{itemize}
\end{slide}

\begin{slide}{Code}
\begin{itemize}
  \item Code below. Note format for reading tab-separated data.
\begin{verbatim}
options linesize=70;

data jumping;
  infile "jumping.dat" delimiter='09'x;
  input group $ g density;

proc means;
  var density;
  class group;

proc glm;
  class group;
  model density=group;
  means group / tukey;
  means group / bon;

\end{verbatim}


  \end{itemize}
  
\end{slide}

\begin{slide}{Comments}

  \begin{itemize}
  \item ``Straightforward'' one-way ANOVA.
  \item Get table of group means and SDs. Assumption: population SD in each group the same, so sample SDs should be ``not too different''.
  \item Tukey's method asks: ``how far apart might lowest and highest sample group means be, if population means all same?''. Anything larger than that declared significantly different.
  \item Bonferroni's method allows for number of paired comparisons, in general for $n$ groups is $n(n-1)/2$, here 3: divide $\alpha$ by 3 for each test (eg.\ $0.05/3=0.0167$). More ``conservative'' than Tukey.
  \end{itemize}
  
\end{slide}

\begin{slide}{Output part 1}

{\scriptsize
\begin{verbatim}
                      Analysis Variable : density
             N
 group     Obs   N          Mean       Std Dev       Minimum       Maximum
 -------------------------------------------------------------------------
 Control    10  10   601.1000000    27.3636011   554.0000000   653.0000000
 Highjump   10  10   638.7000000    16.5935061   622.0000000   674.0000000
 Lowjump    10  10   612.5000000    19.3290225   588.0000000   638.0000000
 -------------------------------------------------------------------------

                          Dependent Variable: density   

                                     Sum of
 Source                    DF       Squares   Mean Square  F Value  Pr > F
 Model                      2    7433.86667    3716.93333     7.98  0.0019
 Error                     27   12579.50000     465.90741                 
 Corrected Total           29   20013.36667                               

 Source                    DF     Type I SS   Mean Square  F Value  Pr > F
 group                      2   7433.866667   3716.933333     7.98  0.0019

 Source                    DF   Type III SS   Mean Square  F Value  Pr > F
 group                      2   7433.866667   3716.933333     7.98  0.0019

\end{verbatim}
}

\end{slide}

\begin{slide}{Notes}

  \begin{itemize}
  \item 
Sample SDs not too different. (Argue that rats were randomly assigned to groups, so population SDs necessarily same.)

\item $F$-tests for model as a whole and for groups (same)
  significant: there is effect of jumping on bone density. Use
  multiple comparisons to see what: Tukey then Bonferroni.
  \end{itemize}
  
\end{slide}

\begin{slide}{Tukey}
  
{\scriptsize
\begin{verbatim}
             Tukey's Studentized Range (HSD) Test for density

               Minimum Significant Difference        23.934

        Means with the same letter are not significantly different.

       Tukey Grouping          Mean      N    group

                    A       638.700     10    Highjump

                    B       612.500     10    Lowjump
                    B
                    B       601.100     10    Control

\end{verbatim}
}

High jumping has a significantly different (better) effect on bone density; no signficant difference between low jumping and control.


\end{slide}

\begin{slide}{Bonferroni}

{\scriptsize
\begin{verbatim}
                   Bonferroni (Dunn) t Tests for density

                  Minimum Significant Difference   24.639

        Means with the same letter are not significantly different.
 
         Bon Grouping          Mean      N    group

                    A       638.700     10    Highjump
                                                      
                    B       612.500     10    Lowjump 
                    B                                 
                    B       601.100     10    Control 

\end{verbatim}
}

\begin{itemize}
\item Here, same conclusions as before.
\item But note min sig difference, 24.639, larger than Tukey (23.934).
\item Bonferroni has harder job finding significant differences if they exist.
\end{itemize}
  
\end{slide}

\begin{slide}{Another example: scaffolds}

  \begin{itemize}
  \item Repair serious wounds by inserting material as ``scaffold'' for body's repair cells to use as template for new tissue.
  \item Scaffolds made from extracellular material (ECMs) promising (made from biological material).
  \item Study: use mice to compare effects of 6 types of material.
  \item Response: \% glucose phosphated isomerase (GPI) cells in region of wound: higher better.
  \item GPI measured 2, 4, 8 weeks after tissue repair.
  \item 3 mice for each combo of material (6) and weeks (3): 54 total.
  \item Data: material, weeks, GPI.
  \item See whether GPI depends on either/both of material and weeks or their interaction.
  \end{itemize}

  
\end{slide}

\begin{slide}{Data }

\begin{verbatim}
ecm1  2   70
ecm1  2   75
ecm1  2   65
ecm1  4   55
ecm1  4   70
ecm1  4   70
ecm1  8   60
ecm1  8   65
ecm1  8   65
ecm2  2   60
...
mat3  8   5
mat3  8   15
mat3  8   10

\end{verbatim}
  
\end{slide}

\begin{slide}{Code}

\begin{verbatim}
options linesize=75;

data scaffold;
  infile "scaffold.dat";
  input material $ weeks gpi;

proc glm;
  class material weeks;
  model gpi=material|weeks;
\end{verbatim}

  \begin{itemize}
  \item 
Declare ``weeks'' as a categorical variable too (look for any differences among weeks), then fit model saying GPI depends on both and interaction too.
\item The \verb-|- between material and weeks means ``fit interaction as well as main effects''.

\item (Looking to see whether interaction significant first, then decide what to do next.)
  \end{itemize}
\end{slide}

\begin{slide}{ANOVA output}

{\scriptsize
\begin{verbatim}
                             The GLM Procedure
Dependent Variable: gpi   
                                     Sum of
 Source                    DF       Squares   Mean Square  F Value  Pr > F

 Model                     17   37609.25926    2212.30937    86.88  <.0001
 Error                     36     916.66667      25.46296                 
 Corrected Total           53   38525.92593                               
...
 Source                    DF     Type I SS   Mean Square  F Value  Pr > F
 material                   5   35659.25926    7131.85185   280.09  <.0001
 weeks                      2     867.59259     433.79630    17.04  <.0001
 material*weeks            10    1082.40741     108.24074     4.25  0.0006

 Source                    DF   Type III SS   Mean Square  F Value  Pr > F
 material                   5   35659.25926    7131.85185   280.09  <.0001
 weeks                      2     867.59259     433.79630    17.04  <.0001
 material*weeks            10    1082.40741     108.24074     4.25  0.0006

\end{verbatim}
}

Look at interaction test (bottom line) first: significant, so don't do any other tests. GPI depends on weeks in different way according to materials.
  
\end{slide}

\begin{slide}{Doing Tukey for interactions}

Requires a trick: make new variable that is material-week combination, then do {\em 1}-way ANOVA on that, looking only at Tukey output:

\begin{verbatim}
data scaffold;
  infile "scaffold.dat";
  input material $ weeks gpi;
  mw=cat(material,"-",weeks);

proc glm;
  class mw;
  model gpi=mw;
  means mw / tukey;

\end{verbatim}

If you check, ``model SS'' same for this analysis as for original one.
  
\end{slide}

\begin{slide}{Tukey output}

{\scriptsize
\begin{verbatim}
           Tukey Grouping          Mean      N    mw

                   A             73.333      3    ecm3    -8
                   A             73.333      3    ecm3    -4
                   A             71.667      3    ecm3    -2
                   A             70.000      3    ecm1    -2
                   A             65.000      3    ecm1    -4
                   A             65.000      3    ecm2    -2
              B    A             63.333      3    ecm1    -8
              B    A             63.333      3    ecm2    -8
              B    A             63.333      3    ecm2    -4
              B                  48.333      3    mat1    -2
                   C             26.667      3    mat3    -2
              D    C             23.333      3    mat1    -4
              D    C    E        21.667      3    mat1    -8
              D    C    E        11.667      3    mat3    -4
              D         E        10.000      3    mat2    -2
              D         E        10.000      3    mat3    -8
                        E         6.667      3    mat2    -8
                        E         6.667      3    mat2    -4
\end{verbatim}
}
  
\end{slide}

\begin{slide}{Interpretation}

  \begin{itemize}
  \item Complicated, because of overlapping lines.
  \item No sig.\ differences among ECMs.
  \item ECMs all better than MATs except mat1 at 2 weeks.
  \item Other MATs worse, with complicated pattern of significant differences.
  \item No consistent pattern of which \#weeks best for each material (explains significant interaction).
  \item Next step should be: MAT materials no good, so do another experiment on just ECMs.
  \item We cheat --- extract data for just ECMs!
  \end{itemize}
  
\end{slide}

\begin{slide}{Just the ECMs: code}

First do the same analysis again, checking for significant interaction:

\begin{verbatim}
data scaffold;
  infile "scaffold2.dat";
  input material $ weeks gpi;

proc glm;
  class material weeks;
  model gpi=weeks|material;
\end{verbatim}
  
\end{slide}

\begin{slide}{Interaction test}

{\scriptsize
\begin{verbatim}
                             The GLM Procedure
 
Dependent Variable: gpi   
                                     Sum of
 Source                    DF       Squares   Mean Square  F Value  Pr > F

 Model                      8    468.518519     58.564815     1.62  0.1874
 Error                     18    650.000000     36.111111                 
 Corrected Total           26   1118.518519                               

 Source                    DF     Type I SS   Mean Square  F Value  Pr > F
 weeks                      2    24.0740741    12.0370370     0.33  0.7209
 material                   2   385.1851852   192.5925926     5.33  0.0152
 material*weeks             4    59.2592593    14.8148148     0.41  0.7989

 Source                    DF   Type III SS   Mean Square  F Value  Pr > F
 weeks                      2    24.0740741    12.0370370     0.33  0.7209
 material                   2   385.1851852   192.5925926     5.33  0.0152
 material*weeks             4    59.2592593    14.8148148     0.41  0.7989

\end{verbatim}
}

No significant interaction (very bottom line), so re-run analysis without (and do Tukey accordingly).
  
\end{slide}

\begin{slide}{Revised code}
Read data as before, and then this:

\begin{verbatim}
proc glm;
  class material weeks;
  model gpi=weeks material;
  means material weeks / tukey;
\end{verbatim}

\begin{itemize}
\item 
Note lack of \verb-|- in \verb-model- line, and back to regular Tukey.
\item No interaction means effect of weeks on GPI same for each material, and effect of material on GPI same for each number of weeks.
\item So get separate Tukeys to see which materials best, which \#weeks best.
\end{itemize}
  
\end{slide}

\begin{slide}{The ANOVA}

{\scriptsize
\begin{verbatim}
                             The GLM Procedure
 
Dependent Variable: gpi   
                                     Sum of
 Source                    DF       Squares   Mean Square  F Value  Pr > F

 Model                      4    409.259259    102.314815     3.17  0.0335
 Error                     22    709.259259     32.239057                 
 Corrected Total           26   1118.518519                               

 Source                    DF     Type I SS   Mean Square  F Value  Pr > F
 weeks                      2    24.0740741    12.0370370     0.37  0.6927
 material                   2   385.1851852   192.5925926     5.97  0.0085

 Source                    DF   Type III SS   Mean Square  F Value  Pr > F
 weeks                      2    24.0740741    12.0370370     0.37  0.6927
 material                   2   385.1851852   192.5925926     5.97  0.0085

\end{verbatim}
}

Significant effect of materials, but not of \#weeks.
  
\end{slide}

\begin{slide}{Tukey}

{\scriptsize
\begin{verbatim}
               Minimum Significant Difference        6.7238

        Means with the same letter are not significantly different.
 
          Tukey Grouping          Mean      N    material

                       A        72.778      9    ecm3    
                       A                                 
                  B    A        66.111      9    ecm1    
                  B                                      
                  B             63.889      9    ecm2    

\end{verbatim}
}
\begin{itemize}
\item Means more than 6.72 different significantly different.
\item ecm3 better than ecm2.
\item ecm1 in curious middle ground: not sig.\ worse than ecm3, not sig.\ better than ecm2.
\item Not enough data to resolve this (ecm1 and ecm3 ``almost'' sig.\ different). 
\end{itemize}

  
\end{slide}

\begin{slide}{Tukey for weeks}

No sig.\ difference due to weeks, so shouldn't really even look at Tukey, but results not surprising:

{\scriptsize
\begin{verbatim}
               Minimum Significant Difference        6.7238

        Means with the same letter are not significantly different.
 
         Tukey Grouping          Mean      N    weeks

                      A        68.889      9    2    
                      A        67.222      9    4    
                      A        66.667      9    8    

\end{verbatim}
}
  
\end{slide}

  




\end{document}




