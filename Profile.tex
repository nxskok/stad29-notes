\documentclass[pdf]{prosper}
\usepackage[Lakar]{HA-prosper}
\usepackage{graphicx}

\begin{document}

\begin{slide}{Repeated measures by profile analysis}

  \begin{itemize}
  \item More than one response {\em measurement} for each subject. Might be
    \begin{itemize}
    \item measurements of the same thing at different times
    \item measurements of different (but related) things
    \end{itemize}
  \item Variation: each subject does several different treatments at different times (called {\em crossover design}).
  \item Expect measurements on same subject to be correlated, so
    assumptions of independence will fail.
  \item Called {\em repeated measures}. Different approaches, but {\em profile analysis} uses PROC GLM and looks like MANOVA.
  \end{itemize}
\end{slide}

  \begin{slide}{Some fake data}

    \begin{itemize}
    \item Here are some data I made up:

\begin{verbatim}
a 10 10 9 10
a 11 9 10 11
a 10 11 10 9
b 9 10 12 10
b 11 10 10 8
b 11 10 8 9
\end{verbatim}


    \item 6 subjects; 2 treatments A and B, 4 (repeated) measurements of some response (at 4 different times).
    \item Nothing much happening:
      \begin{itemize}
      \item no difference between the treatments (no treatment effect)
      \item no trend over time (values just ``jumping about randomly'' for each subject).
      \end{itemize}
    \item Expect to see no significant test results.
    \item Imagine plotting mean response ($y$-axis) vs.\ time ($x$-axis), labelling response by treatment --- ``profile''.


    \end{itemize}
    
  \end{slide}

  \begin{slide}{Doing a repeated measures analysis}

\begin{verbatim}
data rm;
  infile "rm1.dat";
  input trt $ y1 y2 y3 y4;

proc glm;
  class trt;
  model y1 y2 y3 y4 = trt / nouni;
  repeated time;

\end{verbatim}

    \begin{itemize}
    \item In ``model'', put the multiple responses to left of =, like MANOVA.
    \item \verb-nouni- suppresses univariate ANOVAs (not valid/helpful anyway).
    \item specify that the 4 responses are measurements at different times.
    \item Output contains 2 MANOVAs and a univariate ANOVA.
    \end{itemize}
    
  \end{slide}

  \begin{slide}{Output for the first analysis}
    {\scriptsize
\begin{verbatim}
                    Repeated Measures Level Information
 
         Dependent Variable          y1       y2       y3       y4
              Level of time           1        2        3        4

                MANOVA Test Criteria and Exact F Statistics
                   for the Hypothesis of no time Effect
 
Statistic                       Value   F Value   Num DF   Den DF   Pr > F
Wilks' Lambda              0.60922541      0.43        3        2   0.7557
Pillai's Trace             0.39077459      0.43        3        2   0.7557
Hotelling-Lawley Trace     0.64142857      0.43        3        2   0.7557
Roy's Greatest Root        0.64142857      0.43        3        2   0.7557
\end{verbatim}
}

\begin{itemize}
\item No trend over time for either treatment. (No evidence that mean responses at different times are different.)
\item Next test time by treatment interaction, also
  non-significant: no overall difference in response over times, so
  that non-pattern must be same for both treatment groups.
\end{itemize}
    
  \end{slide}

  \begin{slide}{Last ANOVA for first data set}

{\scriptsize
\begin{verbatim}
                             The GLM Procedure
                  Repeated Measures Analysis of Variance
             Tests of Hypotheses for Between Subjects Effects

 Source                    DF   Type III SS   Mean Square  F Value  Pr > F

 trt                        1    0.16666667    0.16666667     0.40  0.5614
 Error                      4    1.66666667    0.41666667                 

\end{verbatim}
}

This tests whether there is a treatment effect, by comparing mean of
the 4 response variables for the treatment groups (so is ordinary
ANOVA). Not significant either.

\vspace{3ex}

Next, change the data to produce a treatment effect but still no time trend:
    
  \end{slide}

  \begin{slide}{Data set 2}

\begin{verbatim}
a 10 10 9 10
a 11 9 10 11
a 10 11 10 9
b 11 10 13 11
b 14 12 12 11
b 15 13 9 11
\end{verbatim}

    \begin{itemize}
    \item Now treatment B looks to have a slightly higher mean, so we might find a significant treatment effect.
    \item Still no apparent differences between times, same for each treatment.
    \item Run same code on this data set (changing only name of data file).
    \end{itemize}


  \end{slide}

  \begin{slide}{MANOVAs for data set 2}

{\scriptsize
\begin{verbatim}
                MANOVA Test Criteria and Exact F Statistics
                   for the Hypothesis of no time Effect

Statistic                       Value   F Value   Num DF   Den DF   Pr > F
Wilks' Lambda              0.17789982      3.08        3        2   0.2546
Pillai's Trace             0.82210018      3.08        3        2   0.2546
Hotelling-Lawley Trace     4.62114125      3.08        3        2   0.2546
Roy's Greatest Root        4.62114125      3.08        3        2   0.2546

                MANOVA Test Criteria and Exact F Statistics
                 for the Hypothesis of no time*trt Effect
 
Statistic                       Value   F Value   Num DF   Den DF   Pr > F
Wilks' Lambda              0.23153563      2.21        3        2   0.3263
Pillai's Trace             0.76846437      2.21        3        2   0.3263
Hotelling-Lawley Trace     3.31898971      2.21        3        2   0.3263
Roy's Greatest Root        3.31898971      2.21        3        2   0.3263

\end{verbatim}
}

No significant difference between times (or difference in pattern of responses over time for the treatments. As we guessed.
    
  \end{slide}

  \begin{slide}{Between-subjects analysis for data set 2}

{\scriptsize
\begin{verbatim}
                            The GLM Procedure
                  Repeated Measures Analysis of Variance
             Tests of Hypotheses for Between Subjects Effects

 Source                    DF   Type III SS   Mean Square  F Value  Pr > F

 trt                        1   20.16666667   20.16666667    30.25  0.0053
 Error                      4    2.66666667    0.66666667                 

\end{verbatim}
}

Treatment effect we introduced is indeed significant.

\vspace{1ex}

\end{slide}

\begin{slide}{Introducing a time effect}

Now make another change to data:

\begin{verbatim}
a 10 10 11 13
a 11 9 12 14
a 10 11 12 12
b 11 10 15 15
b 10 12 14 14
b 12 13 13 15
\end{verbatim}

This time responses at times 3 and 4 seem higher, so expect a time effect now. But pattern of responses over time still same for both treatments, so don't expect a treatment-by-time interaction.

\vspace{3ex}

Run the same code again.
    
  \end{slide}

  \begin{slide}{MANOVAs for data set 3}

{\scriptsize
\begin{verbatim}
                MANOVA Test Criteria and Exact F Statistics
                   for the Hypothesis of no time Effect
 
Statistic                       Value   F Value   Num DF   Den DF   Pr > F
Wilks' Lambda              0.01516477     43.29        3        2   0.0227
Pillai's Trace             0.98483523     43.29        3        2   0.0227
Hotelling-Lawley Trace    64.94230769     43.29        3        2   0.0227
Roy's Greatest Root       64.94230769     43.29        3        2   0.0227

                MANOVA Test Criteria and Exact F Statistics
                 for the Hypothesis of no time*trt Effect
 
Statistic                       Value   F Value   Num DF   Den DF   Pr > F
Wilks' Lambda              0.31515152      1.45        3        2   0.4332
Pillai's Trace             0.68484848      1.45        3        2   0.4332
Hotelling-Lawley Trace     2.17307692      1.45        3        2   0.4332
Roy's Greatest Root        2.17307692      1.45        3        2   0.4332

\end{verbatim}
}

\begin{itemize}
\item Now a significant time effect.
\item Time by treatment interaction still not significant because pattern of change over time same for each treatment.
\end{itemize}
    
  \end{slide}

  \begin{slide}{Still a significant treatment effect}

{\scriptsize
\begin{verbatim}
                             The GLM Procedure
                  Repeated Measures Analysis of Variance
             Tests of Hypotheses for Between Subjects Effects

 Source                    DF   Type III SS   Mean Square  F Value  Pr > F
 trt                        1   15.04166667   15.04166667    36.10  0.0039
 Error                      4    1.66666667    0.41666667                 

\end{verbatim}
}

because Treatment B numbers still bigger than Treatment A.

\end{slide}

\begin{slide}{Finally\ldots}

Make one more change to data:

\begin{verbatim}
a 10 10 14 13
a 11 9 12 14
a 10 11 13 13
b 15 15 11 10 
b 14 14 10 12
b 13 15 10 11 

\end{verbatim}

\begin{itemize}
\item Now the time 3 and 4 numbers are bigger for treatment A and
  smaller for treatment B. 
\item Effect of time, but 
  different for each treatment. 
\item So now time by treatment
  interaction should be significant.

\end{itemize}

    
  \end{slide}

  \begin{slide}{MANOVAs for data set 4}

{\scriptsize
\begin{verbatim}
               MANOVA Test Criteria and Exact F Statistics
                   for the Hypothesis of no time Effect
 Statistic                       Value   F Value   Num DF   Den DF   Pr > F
Wilks' Lambda              0.44926108      0.82        3        2   0.5913
Pillai's Trace             0.55073892      0.82        3        2   0.5913
Hotelling-Lawley Trace     1.22587719      0.82        3        2   0.5913
Roy's Greatest Root        1.22587719      0.82        3        2   0.5913

                MANOVA Test Criteria and Exact F Statistics
                 for the Hypothesis of no time*trt Effect
Statistic                       Value   F Value   Num DF   Den DF   Pr > F
Wilks' Lambda              0.01797044     36.43        3        2   0.0268
Pillai's Trace             0.98202956     36.43        3        2   0.0268
Hotelling-Lawley Trace    54.64692982     36.43        3        2   0.0268
Roy's Greatest Root       54.64692982     36.43        3        2   0.0268

\end{verbatim}
}

\begin{itemize}
\item Interaction indeed significant: pattern of change over time depends on treatment.
\item Main effect not significant because mean scores for each time (over all the data) aren't very different.
\end{itemize}
    
  \end{slide}


  \begin{slide}{There is still a treatment effect}

{\scriptsize
\begin{verbatim}
                             The GLM Procedure
                  Repeated Measures Analysis of Variance
             Tests of Hypotheses for Between Subjects Effects

 Source                    DF   Type III SS   Mean Square  F Value  Pr > F

 trt                        1    4.16666667    4.16666667    25.00  0.0075
 Error                      4    0.66666667    0.16666667                 

\end{verbatim}
}

\end{slide}

\begin{slide}{In summary}

  \begin{itemize}
  \item 
Hard to understand what all the tests are showing, so manipulated data
to produce results we could guess (for easier understanding).
\item Test of time effect called test for ``flatness'' of profiles.
\item Test of time by treatment(s) interaction called test of ``parallelism'' of profiles.
\item Test of treatment effects called test of ``levels''.
  \end{itemize}
    
  \end{slide}


\begin{slide}{A more realistic example}

  \begin{itemize}
  \item Do subjects from different professions differ in what they think about different leisure activities?
  \item 3 occupational groups, bellydancers, politicians and administrators; 5 subjects from each group.
  \item Each subject participates in 4 activities, reading, dancing, TV-watching, skiing; rates satisfaction with each on 10-point scale.
  \item Data like this. (Scores on activities as listed.)
{\scriptsize
\begin{verbatim}
bellydancer 7 10 6 5
bellydancer 8  9 5 7
bellydancer 5 10 5 8
politician  4  4 4 4
politician  6  4 5 3
politician  5  5 5 6
admin       3  1 1 2
admin       5  3 1 5
admin       4  2 2 5
\end{verbatim}
}
\item Profession group plays role of treatment, activity plays role of time.
  \end{itemize}
  
\end{slide}

\begin{slide}{Some means}

  \begin{tabular}{|l|rrrr|r|}
    \hline
    Group & Reading & Dancing & TV & Skiing & Activities\\
    \hline
    Bellydancers & 6.6 & 9.4 & 5.8 & 7.4 & 7.3\\
    Politicians & 5.0 & 4.8 & 5.2 & 5.3 & 5.0\\
    Administrators & 5.0 & 2.0 & 1.8 & 3.8 & 3.2\\
    \hline
    Groups & 5.3 & 5.4 & 4.3 & 5.4 & 5.2\\
    \hline
  \end{tabular}
  
  \vspace{3ex}

  \begin{itemize}
  \item Mean scores for each activity overall quite similar.
  \item Mean scores for each profession group very different.
  \item Bellydancers like dancing; administrators hate everything but
    reading.
  \item Are any of these differences significant?
  \end{itemize}
  
\end{slide}

\begin{slide}{Repeated measures code}

  \begin{itemize}
  \item Code:

\begin{verbatim}
options linesize=75;
data profile;
  infile "profile.dat";
  input group $ read dance tv ski;
proc glm;
  class group;
  model read dance tv ski = group / nouni;
  repeated activity;

\end{verbatim}
  \item \verb-group- is profession group.
  \item ``repeated'' line says that the responses are all ``activities''.
  \item ``Nouni'': omit separate 1-way analyses by activity.
\end{itemize}
  
\end{slide}

\begin{slide}{Output (edited)}

{\scriptsize
\begin{verbatim}
                MANOVA Test Criteria and Exact F Statistics
                 for the Hypothesis of no activity Effect
 
Statistic                       Value   F Value   Num DF   Den DF   Pr > F
Wilks' Lambda              0.27913735      8.61        3       10   0.0040
Pillai's Trace             0.72086265      8.61        3       10   0.0040
Hotelling-Lawley Trace     2.58246571      8.61        3       10   0.0040
Roy's Greatest Root        2.58246571      8.61        3       10   0.0040

              MANOVA Test Criteria and F Approximations for
                the Hypothesis of no activity*group Effect
 
Statistic                       Value   F Value   Num DF   Den DF   Pr > F
Wilks' Lambda              0.07627855      8.74        6       20   <.0001
Pillai's Trace             1.43341443      9.28        6       22   <.0001
Hotelling-Lawley Trace     5.42784967      8.73        6   11.714   0.0009
Roy's Greatest Root        3.54059987     12.98        3       11   0.0006

      NOTE: F Statistic for Roy's Greatest Root is an upper bound.
              NOTE: F Statistic for Wilks' Lambda is exact.
\end{verbatim}
}

\end{slide}

\begin{slide}{Output part 2}


\begin{itemize}
\item Significant difference in mean scores (for all the subjects) over activities, even though overall means were not that different.
\item The pattern of scores over activities is definitely different for each profession group.
\end{itemize}




{\scriptsize
\begin{verbatim}
                 Repeated Measures Analysis of Variance
             Tests of Hypotheses for Between Subjects Effects

 Source                    DF   Type III SS   Mean Square  F Value  Pr > F

 group                      2   172.9000000    86.4500000    44.14  <.0001
 Error                     12    23.5000000     1.9583333                 

\end{verbatim}
}

\begin{itemize}

\item Those different mean scores (over activities) for each profession are very clearly significantly different.

\end{itemize}
  
\end{slide}

\begin{slide}{Another example: histamine in dogs}
  
  \begin{itemize}
  \item 8 dogs take part in experiment.
  \item Dogs randomized to one of 2 different drugs.
  \item Response: log of blood concentration of histamine 0, 1, 3 and 5 minutes after taking drug. (Repeated measures.)
  \item Data in dogs2.dat.
  \end{itemize}

\end{slide}

  \begin{slide}{The code}

\begin{verbatim}
options linesize=75;

data dogs;
  infile "dogs2.dat";
  input Drug $ x $ lh1 lh2 lh3 lh4;
  avg=(lh1+lh2+lh3+lh4)/4;

proc glm;
  class Drug;
  model lh1 lh2 lh3 lh4 = Drug / nouni;
  repeated Time;
  lsmeans Drug;

proc glm;
  class Drug;
  model avg=Drug;
  lsmeans Drug;

\end{verbatim}
    
  \end{slide}

  \begin{slide}{Comments on code}
    \begin{itemize}
    \item Calculate mean of 4 responses (\verb-avg-).
    \item Do repeated measures analysis.
    \item \verb-lsmeans- convenient way to get means on 4 variables
      for each Drug.
    \item Also do ordinary ANOVA using average log-histamine level as response, and obtain means.
    \end{itemize}
    
  \end{slide}

  \begin{slide}{Output part 1}

{\scriptsize
\begin{verbatim}
                MANOVA Test Criteria and Exact F Statistics
                   for the Hypothesis of no Time Effect

Statistic                       Value   F Value   Num DF   Den DF   Pr > F
Wilks' Lambda              0.05012095     25.27        3        4   0.0046
Pillai's Trace             0.94987905     25.27        3        4   0.0046
Hotelling-Lawley Trace    18.95173763     25.27        3        4   0.0046
Roy's Greatest Root       18.95173763     25.27        3        4   0.0046

                MANOVA Test Criteria and Exact F Statistics
                 for the Hypothesis of no Time*Drug Effect

Statistic                       Value   F Value   Num DF   Den DF   Pr > F
Wilks' Lambda              0.10523944     11.34        3        4   0.0200
Pillai's Trace             0.89476056     11.34        3        4   0.0200
Hotelling-Lawley Trace     8.50214058     11.34        3        4   0.0200
Roy's Greatest Root        8.50214058     11.34        3        4   0.0200

\end{verbatim}
}
    
  \end{slide}

  \begin{slide}{Comments and drug-effect analysis}

    \begin{itemize}
    \item The histamine levels do change over time, and the pattern of change differs for the 2 drugs (though latter P-value not {\em very} small).
    \item Analysis of drug effect:
    \end{itemize}

{\scriptsize
\begin{verbatim}
                             The GLM Procedure
                  Repeated Measures Analysis of Variance
             Tests of Hypotheses for Between Subjects Effects

 Source                    DF   Type III SS   Mean Square  F Value  Pr > F

 Drug                       1   11.52000000   11.52000000     3.13  0.1274
 Error                      6   22.10263750    3.68377292                 

\end{verbatim}
}

Averaging over time, no significant difference between drugs.

\end{slide}

\begin{slide}{LSMEANS}

{\scriptsize
\begin{verbatim}
                             The GLM Procedure
                            Least Squares Means

 Drug          lh1 LSMEAN      lh2 LSMEAN      lh3 LSMEAN      lh4 LSMEAN
 Morphine     -2.89000000     -1.16000000     -1.99750000     -2.32500000
 Trimetha     -3.02250000      0.13000000     -0.17250000     -0.50750000

\end{verbatim}
}

Both drugs show increase (to time 2) then decrease. (Time effect.) Rate of decrease smaller for Trimetha (time-drug interaction effect).
    
  \end{slide}

  \begin{slide}{The second PROC GLM, edited}

{\scriptsize
\begin{verbatim}
 Source                    DF       Squares   Mean Square  F Value  Pr > F

 Drug                       1    2.88000000    2.88000000     3.13  0.1274
 Error                      6    5.52565938    0.92094323                 
 Corrected Total            7    8.40565938                               

                             The GLM Procedure
                            Least Squares Means

                         Drug          avg LSMEAN
                         Morphine     -2.09312500
                         Trimetha     -0.89312500

\end{verbatim}
}
\begin{itemize}
\item P-value identical to last part of repeated measures analysis.
\item Drug means look different, but not different enough to be significant.
\end{itemize}
    
  \end{slide}

\end{document}