\documentclass[pdf]{prosper}
\usepackage[Lakar]{HA-prosper}
\usepackage{graphicx}


\begin{document}

\begin{slide}{Multi-way frequency analysis}

  \begin{itemize}
  \item A study of gender and eyewear-wearing finds the following frequencies:

    \begin{tabular}{lrrr}
      \hline
      Gender & Contacts & Glasses & None \\
      \hline
      Female & 121 & 32 & 129 \\
      Male & 42 & 37 & 85\\
      \hline
    \end{tabular}


  \item Is there association between eyewear and gender?
  \item Normally answer this with chisquare test (based on observed and expected frequencies from null hypothesis of no association).
  \item Two categorical variables and a frequency.
  \item We assess in way that generalizes to more categorical variables.

  \end{itemize}

\end{slide}

\begin{slide}{Data format}

Data file like this:

\begin{verbatim}
female contacts 121
female glasses   32
female none     129
male   contacts  42
male   glasses   37
male   none      85
\end{verbatim}

as the two categorical variables (gender, type of eyewear) and frequency (number of observations in that category combination).
  
\end{slide}

\begin{slide}{Some code, using PROC CATMOD}

\begin{verbatim}
data lens;
  infile "lenswear.dat";
  input sex $ lenswear $ frequency;

proc catmod; 
  weight frequency;
  model sex*lenswear=_response_;
  loglin sex lenswear sex*lenswear;
\end{verbatim}

\vspace{3ex}

In PROC CATMOD, specify frequency, then SAS black magic to get right thing, then model (on LOGLIN line!).

  
\end{slide}

\begin{slide}{Maximum likelihood analysis}

{\scriptsize
\begin{verbatim}
                      Maximum Likelihood Analysis

              Maximum likelihood computations converged.           

               Maximum Likelihood Analysis of Variance
 
          Source               DF   Chi-Square    Pr > ChiSq
          --------------------------------------------------
          sex                   1        16.10        <.0001
          lenswear              2        64.63        <.0001
          sex*lenswear          2        17.16        0.0002

          Likelihood Ratio      0          .           .    

\end{verbatim}
}

\begin{itemize}
\item Conclude from \verb-sex*lenswear- line that interaction is significant.
\item That is, frequency depends on the sex-lenswear {\em combination} (not just on either variable singly).
\item Or, there is association between sex and lenswear (as usual chisquare test concludes).
\end{itemize}
  
\end{slide}

\begin{slide}{Understanding the association}

{\scriptsize
\begin{verbatim}
              Analysis of Maximum Likelihood Estimates
 
                                                    Standard
         Parameter                       Estimate      Error
         ---------------------------------------------------
         sex          female               0.2217     0.0552
         lenswear     contacts             0.1146     0.0757
                      glasses             -0.6138     0.0889
         sex*lenswear female contacts      0.3074     0.0757
                      female glasses      -0.2943     0.0889

\end{verbatim}
}

Estimates over each variable sum to 0, so complete table as over.
  
\end{slide}

\begin{slide}{All the estimates}

{\scriptsize
\begin{verbatim}
         Parameter                       Estimate      Error
         ---------------------------------------------------
         sex          female               0.2217     0.0552
                      male                -0.2217
         lenswear     contacts             0.1146     0.0757
                      glasses             -0.6138     0.0889
                      none                 0.4992
         sex*lenswear female contacts      0.3074     0.0757
                      female glasses      -0.2943     0.0889
                      female none         -0.0131
                      male contacts       -0.3074
                      male glasses         0.2943
                      male none            0.0131

\end{verbatim}
}

\begin{itemize}
\item Look for large (plus or minus) estimates.
\item Females more likely to wear contacts and males glasses than expected (if no association).
\item Overall, more females in study, and people less likely to wear glasses than other types of eyewear (and most likely to wear none).
\end{itemize}
  
\end{slide}

\begin{slide}{Another example: reading, gender and occupation}

  \begin{tabular}{|l|l|rr|r|}
    \hline
    && \multicolumn{2}{|c|}{Preferred reading} &\\
    Profession & Sex & Scifi & Spy & Total \\
    \hline
    Politician & Male & 15 & 15 & 30\\
    & Female & 10 & 15 & 25\\
    \hline
    & Total & 25 & 30 & 55\\
    \hline \hline
    Administrator & Male & 10 & 30 & 40\\
    & Female & 5 & 10 & 15\\
    \hline
    & Total & 15 & 40 & 55\\
    \hline \hline
    Bellydancer & Male & 5 & 5 & 10\\
    & Female & 10 & 25 & 35 \\
    \hline
    & Total & 15 & 30 & 45\\
    \hline
    \hline
  \end{tabular}

  Altogether 80 males and 75 females.
  
\end{slide}

\begin{slide}{Data into SAS}

  This time there are 3 categorical variables (profession, sex,
  preferred reading) and a frequency. Arrange with one frequency on
  each line (without totals):

\begin{verbatim}
politician male scifi 15
politician male spy 15
politician female scifi 10
politician female spy 15
administrator male scifi 10
administrator male spy 30
administrator female scifi 5
administrator female spy 10
bellydancer male scifi 5
bellydancer male spy 5
bellydancer female scifi 10
bellydancer female spy 25
\end{verbatim}
  
\end{slide}

\begin{slide}{The code}

\begin{verbatim}
data small;
  infile "multiway.dat";
  input profession $ sex $ readtype $ freq;

proc catmod;
  weight freq;
  model profession*sex*readtype=_response_;
  loglin profession sex readtype profession*sex 
    profession*readtype sex*readtype 
    profession*sex*readtype;
\end{verbatim}

\vspace{3ex}

  Loglin line could have been written \verb-profession|sex|readtype-
  (include main effects and all interactions between variables), but
  done this way for a reason.
  
\end{slide}

\begin{slide}{Assessing what to take out}

From the ``maximum likelihood analysis of variance'':

{\scriptsize
\begin{verbatim}
           Maximum Likelihood Analysis of Variance

  Source                      DF   Chi-Square    Pr > ChiSq

  profession                   2         3.46        0.1777
  sex                          1         0.01        0.9256
  readtype                     1         7.61        0.0058
  profession*sex               2        17.58        0.0002
  profession*readtype          2         2.62        0.2691
  sex*readtype                 1         0.66        0.4168
  profession*sex*readtype      2         1.89        0.3894

  Likelihood Ratio             0          .           .

\end{verbatim}
}

\begin{itemize}
\item Model fits perfectly (see Likelihood Ratio line)
\item As ANOVA, remove 3-way interaction.
\item Change \verb-loglin- line to this:

\begin{verbatim}
  loglin profession sex readtype profession*sex 
    profession*readtype sex*readtype;
\end{verbatim}
\end{itemize}
  
\end{slide}

\begin{slide}{Output from this}

{\scriptsize
\begin{verbatim}
          Maximum Likelihood Analysis of Variance

   Source                  DF   Chi-Square    Pr > ChiSq

   profession               2         3.58        0.1674
   sex                      1         0.00        0.9453
   readtype                 1        13.02        0.0003
   profession*sex           2        23.00        <.0001
   profession*readtype      2         4.32        0.1155
   sex*readtype             1         0.62        0.4321

   Likelihood Ratio         2         1.85        0.3969

\end{verbatim}
}

\begin{itemize}
\item Bottom line: ``is there evidence of lack of fit?'' Answer no: model fits OK.
\item Now look at two-way interactions and take out non-significant ones.
\item Code for that:

\begin{verbatim}
  loglin profession sex readtype profession*sex;
\end{verbatim}
\end{itemize}
  
\end{slide}

\begin{slide}{Output}

{\scriptsize
\begin{verbatim}
        Maximum Likelihood Analysis of Variance

    Source               DF   Chi-Square    Pr > ChiSq

    profession            2         2.90        0.2348
    sex                   1         0.03        0.8686
    readtype              1        12.68        0.0004
    profession*sex        2        22.79        <.0001

    Likelihood Ratio      5         6.56        0.2557

\end{verbatim}
}

\begin{itemize}
\item Model still fits OK (last line).
\item Two-way interaction significant: stays.
\item Main effects involving profession and sex have to stay.
\item Main effect involving reading type significant, so stays.
\item Done. Now interpret the estimates.
\end{itemize}
  
\end{slide}

\begin{slide}{The maximum likelihood estimates}

with missing ones filled in:

{\scriptsize
\begin{verbatim}
                                              Standard    Chi-
 Parameter                         Estimate      Error  Square Pr > ChiSq

 profession     administ             0.0526     0.1257    0.18     0.6753
                bellydan            -0.2169     0.1374    2.49     0.1144
 sex            female               0.0149     0.0903    0.03     0.8686
 readtype       scifi               -0.2989     0.0839   12.68     0.0004
                spy                  0.2989
 profession*sex administ female     -0.5053     0.1257   16.17     <.0001
                bellydan female      0.6114     0.1374   19.82     <.0001
                politician female   -0.1061
                administ male        0.5053
                bellydan male       -0.6114
                politician male      0.1061

\end{verbatim}
}

\begin{itemize}
\item Readtype: people overall prefer spy novels
\item Interaction: bellydancers tend to be female and administrators male (more so than even split of males/females would suggest).
\end{itemize}
  
\end{slide}

\begin{slide}{A different way to read the data}

  \begin{itemize}
  \item Entering the words into the data file is repetitive. Start with data as laid out in table (in \verb-freq.dat-):

{\scriptsize
\begin{verbatim}
15 15
10 15
10 30
5 10
5 5
10 25
\end{verbatim}
}


\item Then use ``loops'' to associate with variables:

{\scriptsize
\begin{verbatim}
data myfreq;
  infile "freq.dat";
  do profession="politician   ","administrator","bellydancer";
    do sex="male  ","female";
      do readtype="scifi","spy";
        input freq @@;
        output;
      end;
    end;
  end;

\end{verbatim}
}

\item Resulting data set and PROC CATMOD as before.
    
  \end{itemize}
  
\end{slide}

\begin{slide}{Simpson's paradox: the airlines example}

  \begin{tabular}{|l|rr|rr|}
    \hline
    & \multicolumn{2}{|c|}{Alaska Airlines} & 
    \multicolumn{2}{|c|}{America West}\\
    Airport & On time & Delayed & On time & Delayed\\
    \hline
    Los Angeles & 497 & 62 & 694 & 117\\
    Phoenix & 221 & 12 & 4840 & 415\\
    San Diego & 212 & 20 & 383 & 65\\
    San Francisco & 503 & 102 & 320 & 129 \\
    Seattle & 1841 & 305 & 201 & 61\\
    \hline
    Total & 3274 & 501 & 6438 & 787\\
    \hline
  \end{tabular}

  \begin{itemize}
  \item Alaska: 13.3\% flights delayed ($501/(3274+501)$).
  \item America West: 10.9\% ($787/(6438+787)$).
  \item America West more punctual, right?
  \end{itemize}
  
\end{slide}

\begin{slide}{Percentage delayed by airport}

  \begin{tabular}{|l|rr|}
    \hline
    Airport & Alaska & America West\\
    \hline
    Los Angeles & 11.4 & 14.4\\
    Phoenix & 5.2 & 7.9\\
    San Diego & 8.6 & 14.5\\
    San Francisco & 16.9 & 28.7\\
    Seattle & 14.2 & 23.2 \\
    \hline
    Total & 13.3 & 10.9 \\
    \hline
    
  \end{tabular}

  \begin{itemize}
  \item America West better overall, yet {\em worse at every single airport}!
  \item Can PROC CATMOD explain?
  \item 3 categorical variables (airline, airport, on time/delayed), frequency.
  \end{itemize}
  
\end{slide}

\begin{slide}{Data for SAS}

\begin{verbatim}
losangeles alaska ontime 497
losangeles alaska delayed 62
losangeles aw ontime 694
losangeles aw delayed 117
phoenix alaska ontime 221
phoenix alaska delayed 12
phoenix aw ontime 4840
phoenix aw delayed 415
...
sanfran alaska ontime 503
sanfran alaska delayed 102
sanfran aw ontime 320
sanfran aw delayed 129
seattle alaska ontime 1841
seattle alaska delayed 305
seattle aw ontime 201
seattle aw delayed 61
\end{verbatim}
  
\end{slide}

\begin{slide}{Code}

\begin{verbatim}
data airline;
  infile "airport.dat";
  input airport $ airline $ status $ freq;

proc catmod;
  weight freq;
  model airport*airline*status=_response_;
  loglin airport|airline|status;

\end{verbatim}

Or write out all the effects on the \verb-loglin- line.
  
\end{slide}

\begin{slide}{Alternative form for data}

  \begin{itemize}
  \item Data file:

\begin{verbatim}
497 62 694 117
221 12 4840 415
212 20 383 65
503 102 320 129
1841 305 201 61
\end{verbatim}


  \item Code to read this:

\begin{verbatim}
data myfreq;
  infile "freq2.dat";
  do airport="losangeles  ","phoenix","sandiego",
    "sanfrancisco","seattle";
    do airline="alaska     ","americawest";
      do status="ontime ","delayed";
        input freq @@;
        output;
      end;
    end;
  end;

\end{verbatim}
  \end{itemize}
  
\end{slide}

\begin{slide}{Output}

{\scriptsize
\begin{verbatim}
           Maximum Likelihood Analysis of Variance

   Source                     DF   Chi-Square    Pr > ChiSq

   airport                     4       185.99        <.0001
   airline                     1       118.66        <.0001
   airport*airline             4      1138.97        <.0001
   status                      1      1487.23        <.0001
   airport*status              4        99.56        <.0001
   airline*status              1        29.09        <.0001
   airport*airline*status      4         3.26        0.5156

   Likelihood Ratio            0          .           .

\end{verbatim}
}

\begin{itemize}
\item Complicated model fits perfectly (not interesting)
\item 3-way interaction non-significant: remove.
\item Change loglin line to:

\begin{verbatim}
  loglin airport|airline|status @ 2;

\end{verbatim}
(include all interactions $\le$ 2-way).
\end{itemize}
  
\end{slide}

\begin{slide}{Output now}

{\scriptsize
\begin{verbatim}
        Maximum Likelihood Analysis of Variance

   Source               DF   Chi-Square    Pr > ChiSq

   airport               4       231.19        <.0001
   airline               1       163.72        <.0001
   airport*airline       4      3225.58        <.0001
   status                1      2700.13        <.0001
   airport*status        4       246.27        <.0001
   airline*status        1        41.74        <.0001

   Likelihood Ratio      4         3.22        0.5223

\end{verbatim}
}

\begin{itemize}
\item Model fits OK (no evidence of lack of fit).
\item All 2-way interactions significant: stop here.
\end{itemize}
  
\end{slide}

\begin{slide}{Airline by status, adding missing ones}

{\scriptsize
\begin{verbatim}
                     Analysis of Maximum Likelihood Estimates

                                                Standard        Chi-  Prob >
 Parameter                           Estimate      Error      Square   ChiSq

....
 airline*status  alaska delayed       -0.1361     0.0211       41.74  <.0001
                 alaska ontime         0.1361
                 aw delayed            0.1361
                 aw ontime            -0.1361         
\end{verbatim}
}

\begin{itemize}
\item Alaska {\em more} likely to be on time and America West {\em more} likely to be delayed, allowing for effects of other variables.
\item This in contrast to overall \%'s.
\item Other interactions shed some light.
\end{itemize}
  
\end{slide}

\begin{slide}{Airport by airline}

{\scriptsize
\begin{verbatim}
                     Analysis of Maximum Likelihood Estimates

                                                Standard        Chi-  Prob >
 Parameter                           Estimate      Error      Square   ChiSq
 ....
 airport*airline losangel alaska      -0.0164     0.0261        0.39  0.5303
                 phoenix alaska       -1.4049     0.0302     2165.96  <.0001
                 sandiego alaska      -0.1618     0.0348       21.57  <.0001
                 sanfran alaska        0.3461     0.0287      145.07  <.0001
                 seattle alaska        1.2539
\end{verbatim}
}

\begin{itemize}
\item America West figures negatives of Alaska figures.
\item Frequency less than expected for AA into Phoenix (AA flies less often into Phoenix).
\item Frequency more than expected for AA into San Francisco and Seattle (AA flies more often into San Francisco and Seattle).
\item Conversely, America West flies more into Phoenix and less into San Francisco and Seattle.
\end{itemize}
  
\end{slide}

\begin{slide}{Airport by status}

{\scriptsize
\begin{verbatim}
                     Analysis of Maximum Likelihood Estimates

                                                Standard        Chi-
 Parameter                           Estimate      Error      Square  Pr > ChiSq

 airport*status  losangel delayed     -0.0335     0.0360        0.87  0.3520
                 phoenix delayed      -0.4110     0.0305      181.94  <.0001
                 sandiego delayed     -0.0762     0.0487        2.44  0.1180
                 sanfran delayed       0.3268     0.0343       90.68  <.0001
                 seattle delayed       0.1929
\end{verbatim}
}

\begin{itemize}
\item On-time estimates negatives of delayed figures.
\item Fewer flights to Phoenix are delayed (than to other places).
\item More flights to San Francisco and Seattle delayed.
\end{itemize}

  
\end{slide}

\begin{slide}{Resolution of this Simpson's paradox}

  \begin{itemize}
  \item Alaska Airlines flies mostly into San Francisco and Seattle, while America West flies mostly into Phoenix (airport by airline)
  \item Flights into Phoenix are more likely to be on time, while flights into San Francisco and Seattle are more likely to be delayed.
  \item In ``overall \% late'', AA gets penalized for flying into airports where hard to be on time.
  \item When you allow for who flies where, AA comes out more punctual (as seen in airport-by-airport statistics).
  \end{itemize}
  
\end{slide}

\begin{slide}{Ovarian cancer: a four-way table}

  \begin{itemize}
  \item Retrospective study of ovarian cancer done in 1973.
  \item Information about 299 women operated on for ovarian cancer 10 years previously.
  \item Recorded:
    \begin{itemize}
    \item stage of cancer (early or advanced)
    \item type of operation (radical or limited)
    \item X-ray treatment received (yes or no)
    \item 10-year survival (yes or no)
    \end{itemize}
  \item Survival looks like response (suggests logistic regression). PROC CATMOD finds any associations at all.
  \end{itemize}
  
\end{slide}

\begin{slide}{The data}

for SAS purposes:

{\scriptsize
\begin{verbatim}
early radical no no 10
early radical no yes 41
early radical yes no 17
early radical yes yes 64
early limited no no 1
early limited no yes 13
early limited yes no 3
early limited yes yes 9
advanced radical no no 38
advanced radical no yes 6
advanced radical yes no 64
advanced radical yes yes 11
advanced limited no no 3
advanced limited no yes 1
advanced limited yes no 13
advanced limited yes yes 5

\end{verbatim}
}

Stage, type, x-ray, survival, frequency.
  
\end{slide}

\begin{slide}{The code}

hopefully looking familiar by now:

\begin{verbatim}
data cancer;
  infile "cancer.dat";
  input stage $ operation $ xray $ survival $ count;

proc catmod;
  weight count;
  model stage*operation*xray*survival=_response_;
  loglin stage|operation|xray|survival;

\end{verbatim}
  
\end{slide}

\begin{slide}{Alternative data entry}

  \begin{itemize}
  \item Data like this:
\begin{verbatim}
10 41 17 64 1 13 3 9
38 6 64 11 3 1 13 5
\end{verbatim}
  \item All values for each stage first. Within each stage, all values for kind of operation; within these, all values for X-ray, then all values for survival:

{\scriptsize
\begin{verbatim}
data freq;
  infile "freq3.dat";
  do stage="early    ","advanced";
    do operation="radical","limited";
      do xray="no ","yes";
        do survival="no ","yes";
          input count @@;
          output;
        end;
      end;
    end;
  end;       
\end{verbatim}
}
  \end{itemize}
  
\end{slide}

\begin{slide}{Output \#1}

{\scriptsize
\begin{verbatim}
            Maximum Likelihood Analysis of Variance
   Source                       DF   Chi-Square    Pr > ChiSq
   operation*xray                1         0.80        0.3712
   stage*operation*xray          1         1.33        0.2495
   survival                      1         0.15        0.6979
   stage*survival                1        40.09        <.0001
   operation*survival            1         1.69        0.1930
   stage*operation*survival      1         0.11        0.7425
   xray*survival                 1         0.48        0.4871
   stage*xray*survival           1         0.87        0.3502
   operation*xray*survival       1         0.48        0.4874
   stage*operat*xray*surviv      1         0.57        0.4499

   Likelihood Ratio              0          .           .
\end{verbatim}
}

\begin{itemize}
\item Four-way interaction and all 3-way interactions not significant: remove all, and check resulting model for fit.
\item Change loglin line to this:

\begin{verbatim}
loglin stage|operation|xray|survival @ 2;
\end{verbatim}

that is, keep main effects and interactions up to 2-way.
\end{itemize}
  
\end{slide}

\begin{slide}{Output \#2}

{\scriptsize
\begin{verbatim}
        Maximum Likelihood Analysis of Variance
  Source                 DF   Chi-Square    Pr > ChiSq
  stage                   1         0.27        0.6033
  operation               1       102.15        <.0001
  stage*operation         1         0.59        0.4415
  xray                    1        10.01        0.0016
  stage*xray              1         0.62        0.4324
  operation*xray          1         0.01        0.9326
  survival                1         0.23        0.6294
  stage*survival          1        99.45        <.0001
  operation*survival      1         2.06        0.1511
  xray*survival           1         0.09        0.7696

  Likelihood Ratio        5         7.17        0.2084

\end{verbatim}
}

\begin{itemize}
\item Model still fits all right.
\item Only significant 2-way interaction is stage by survival.
\item Take out others and check fit again.
\item Change loglin line to

{\small
\begin{verbatim}
loglin stage operation xray survival stage*survival;

\end{verbatim}
}
\end{itemize}
  
\end{slide}

\begin{slide}{Output \#3}

{\scriptsize
\begin{verbatim}

        Maximum Likelihood Analysis of Variance

   Source               DF   Chi-Square    Pr > ChiSq

   stage                 1         1.50        0.2202
   operation             1       110.28        <.0001
   xray                  1        17.46        <.0001
   survival              1         0.55        0.4584
   stage*survival        1       100.74        <.0001

   Likelihood Ratio     10        10.99        0.3583

\end{verbatim}
}

\begin{itemize}
\item Model fit still OK (no evidence of lack of fit)
\item Stage and survival main effects have to stay.
\item Operation and X-ray main effects are significant, so they stay.
\item Done. Interpret maximum likelihood estimates.
\end{itemize}
  
\end{slide}

\begin{slide}{Maximum likelihood estimates}
  {\scriptsize
\begin{verbatim}

                  Analysis of Maximum Likelihood Estimates
                                          Standard        Chi-
 Parameter                     Estimate      Error      Square Pr > ChiSq
 stage          advanced        -0.0930     0.0759        1.50    0.2202
 operation      limited         -0.8271     0.0788      110.28    <.0001
 xray           no              -0.2492     0.0596       17.46    <.0001
 survival       no               0.0562     0.0759        0.55    0.4584
 stage*survival advanced no      0.7613     0.0759      100.74    <.0001
\end{verbatim}
}

\begin{itemize}
\item Stage by survival interaction: stage of cancer and survival
  associated. Higher frequency with being in
  advanced stage and not surviving: advanced stage associated
  with non-survival.
\item Fewer women had the limited operation (more had the radical one)
\item Fewer woman had no X-ray treatment (more did have X-ray treatment).
\item Interaction with ``response'' (survival) usually of most interest.
\end{itemize}

\end{slide}

\begin{slide}{General procedure}
\begin{itemize}
\item Start with ``complete model'' including all possible interactions.
\item Look at highest-order interaction(s) remaining, remove if non-significant.
\item If an interaction significant, keep also everything contained within that interaction. Eg.\ A*B interaction significant, keep A and B main effects.
\item Continue until everything either significant or must be kept.
\item Then look at maximum likelihood estimates (can fill in those not shown) and interpret according to whether $+$ or $-$.
  \item Main effects not usually very interesting.
\item Interactions with ``response'' usually of most interest: show association with response.
\end{itemize}
\end{slide}

\end{document}
