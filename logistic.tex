\documentclass[pdf]{prosper}

\usepackage[lakar]{HA-prosper}

\usepackage{graphicx}

% compile with latex and then dvips -t letter

%\setlength{\parindent}{0em}
%\setlength{\parskip}{30pt}

%\setlength{\textwidth}{5in}
%\setlength{\textheight}{5in}

%\setcounter{secnumdepth}{0}


\begin{document}

\begin{slide}{Logistic regression}

  \begin{itemize}
  \item When response variable is measured/counted, regression can work well.
  \item But what if response is yes/no, lived/died/ success/failure?
  \item Model {\em probability} of success.
  \item Probability must be between 0 and 1; need method that ensures this.
  \item {\em Logistic regression} does this; PROC LOGISTIC in SAS.
  \item Begin with simplest case.
  \end{itemize}
  
\end{slide}

\begin{slide}{The rats, part 1}

Rats given dose of some poison; either live or die:

\begin{verbatim}
0 lived
1 died
2 lived
3 lived
4 died
5 died
\end{verbatim}

Basic logistic regression analysis:

{\scriptsize
\begin{verbatim}
options linesize=80;

data rat;
  infile "rat.dat";
  input dose survival $;

proc logistic;
  class survival;
  model survival = dose;
  output out=rat2 pred=pred;

proc print data=rat2;
\end{verbatim}
}  

\end{slide}

\begin{slide}{Output}

{\scriptsize
\begin{verbatim}
                             The LOGISTIC Procedure

                               Model Information

                 Data Set                      WORK.RAT
                 Response Variable             survival
                 Number of Response Levels     2
                 Model                         binary logit
                 Optimization Technique        Fisher's scoring

                    Number of Observations Read           6
                    Number of Observations Used           6

                                Response Profile

                       Ordered                      Total
                         Value     survival     Frequency

                             1     died                 3
                             2     lived                3

                    Probability modeled is survival='died'.

\end{verbatim}
}
  
\end{slide}

\begin{slide}{Output part 2 (edited)}

{\scriptsize
\begin{verbatim}
                            Model Convergence Status

                 Convergence criterion (GCONV=1E-8) satisfied.
... snip
                    Testing Global Null Hypothesis: BETA=0

            Test                 Chi-Square       DF     Pr > ChiSq

            Likelihood Ratio         1.5449        1         0.2139
            Score                    1.4286        1         0.2320
            Wald                     1.2037        1         0.2726

                   Analysis of Maximum Likelihood Estimates

                                     Standard          Wald
      Parameter    DF    Estimate       Error    Chi-Square    Pr > ChiSq

      Intercept     1     -1.6841      1.7978        0.8774        0.3489
      dose          1      0.6736      0.6140        1.2037        0.2726
\end{verbatim}
}
  
\end{slide}

\begin{slide}{Interpreting the output}
  \begin{itemize}
  \item Like (multiple) regression, get:
    \begin{itemize}
    \item overall test of model (``global null hypothesis'')
    \item tests of significance of individual $x$'s (``analysis of
      maximum likelihood estimates'').
    \end{itemize}
  \item     Here none of them significant (only 6 observations).
  \item These tests all agree for regression, but don't for logistic regression. Look for consistent picture (Wald often different from others).
  \item Look at event ``modeled'', here ``died''.
  \item ``Slope'' for dose is positive, meaning that as dose increases, probability of event modelled (death) increases.
  \item Output data set contains predicted probabilities (next slide):
\end{itemize}

\end{slide}

\begin{slide}{Predicted probabilities}

{\scriptsize
\begin{verbatim}
                 Obs    dose    survival    _LEVEL_      pred

                  1       0      lived       died      0.15656
                  2       1      died        died      0.26690
                  3       2      lived       died      0.41658
                  4       3      lived       died      0.58342
                  5       4      died        died      0.73310
                  6       5      died        died      0.84344
\end{verbatim}
}

\vspace{3ex}

``Pred'' is predicted probability of event named by \verb-_LEVEL_- (death). Goes up as dose increases.
  
\end{slide}




\begin{slide}{The rats, part 2}

  \begin{itemize}
  \item More realistic: more rats at each dose (say 10).
  \item Listing each rat on one line makes a big data file.
  \item Use format below: dose, number of deaths, number of trials (rats):
\begin{verbatim}
0 0 10
1 3 10
2 4 10
3 6 10
4 8 10
5 9 10

\end{verbatim}
\item Alter model line for PROC LOGISTIC to say:
\begin{verbatim}
  model deaths/trials = dose;
\end{verbatim}

  \end{itemize}
  
\end{slide}

\begin{slide}{SAS code for this logistic regression}

\begin{verbatim}
options linesize=80;

data rat;
  infile "rat2.dat";
  input dose deaths trials;

proc logistic;
  model deaths/trials = dose;
  output out=rat2 pred=pred lower=lcl upper=ucl;

proc print data=rat2;

\end{verbatim}

\vspace{3ex}

This time, have output data set also contain lower and upper limits of a 95\% CI for each death probability.
  
\end{slide}

\begin{slide}{Output part 1 (edited)}

{\scriptsize
\begin{verbatim}
                    Number of Observations Read           6
                    Number of Observations Used           6
                    Sum of Frequencies Read              60
                    Sum of Frequencies Used              60


                                Response Profile

                       Ordered     Binary           Total
                         Value     Outcome      Frequency

                             1     Event               30
                             2     Nonevent            30


                            Model Convergence Status

                 Convergence criterion (GCONV=1E-8) satisfied.

\end{verbatim}
}

The 6 lines of data correspond to 60 actual rats.
  
\end{slide}

\begin{slide}{Output part 2 (edited)}

{\scriptsize
\begin{verbatim}
                    Testing Global Null Hypothesis: BETA=0

            Test                 Chi-Square       DF     Pr > ChiSq

            Likelihood Ratio        25.0562        1         <.0001
            Score                   21.9657        1         <.0001
            Wald                    16.1449        1         <.0001

                   Analysis of Maximum Likelihood Estimates

                                     Standard          Wald
      Parameter    DF    Estimate       Error    Chi-Square    Pr > ChiSq

      Intercept     1     -2.3619      0.6719       12.3585        0.0004
      dose          1      0.9448      0.2351       16.1449        <.0001

\end{verbatim}
}

\begin{itemize}
\item 
All 4 tests agree: significant effect of dose. 
\item Effect of larger dose is to increase death probability (``slope'' positive).
\end{itemize}
  
\end{slide}

\begin{slide}{Predicted probabilities}

Just run PROC PRINT on output data set:

{\scriptsize
\begin{verbatim}
        Obs    dose    deaths    trials      pred       lcl        ucl

         1       0        0        10      0.08612    0.02463    0.26017
         2       1        3        10      0.19511    0.08646    0.38304
         3       2        4        10      0.38405    0.24041    0.55124
         4       3        6        10      0.61595    0.44876    0.75959
         5       4        8        10      0.80489    0.61696    0.91354
         6       5        9        10      0.91388    0.73983    0.97537

\end{verbatim}
}

\begin{itemize}
\item Predicted death probs increase with dose.
\item Last 2 columns are 95\% CI for prob of death at each dose (eg.\ dose 2, from 0.24 to 0.55).
\item Intervals still quite wide even with $n=60$ rats.
\item Each rat doesn't contribute much information (just lived/died) so need $n$ in hundreds to get precise intervals.
\end{itemize}

  
\end{slide}

\begin{slide}{Multiple logistic regression}

  \begin{itemize}
  \item With more than one $x$, works much like multiple regression.
  \item Example: study of patients with blood poisoning severe enough to warrant surgery. Relate survival to other potential risk factors.
  \item Variables, 1=present, 0=absent:
    \begin{itemize}
    \item survival (death from sepsis=1), response
    \item shock
    \item malnutrition
    \item alcoholism
    \item age (as numerical variable)
    \item bowel infarction
    \end{itemize}
  \item See what relates to death.
  \end{itemize}


  
\end{slide}

\begin{slide}{Some SAS code}

\begin{verbatim}
data x;
  infile "sepsis.dat";
  input death shock malnut alcohol age bowelinf;

proc logistic;
  model death=shock malnut alcohol age bowelinf;
  test malnut=0, bowelinf=0;

proc logistic;
  model death=shock alcohol age bowelinf;
  output out=z pred=p;

proc print data=z;

\end{verbatim}

  Use of PROC LOGISTIC resembles use of PROC REG, including ``test''.
  
\end{slide}

\begin{slide}{Output part 1}

{\scriptsize
\begin{verbatim}
               Number of Observations Used         106

                           Response Profile
 
                  Ordered                      Total
                    Value        death     Frequency
                        1            0            85
                        2            1            21

                   Probability modeled is death=0.

               Testing Global Null Hypothesis: BETA=0
 
       Test                 Chi-Square       DF     Pr > ChiSq

       Likelihood Ratio        52.4060        5         <.0001
       Score                   43.8921        5         <.0001
       Wald                    16.2433        5         0.0062

\end{verbatim}
}

Model as a whole is significant: at least one of the $x$'s helps predict death (actually modelling P(survival)).

\end{slide}

\begin{slide}{Finding significant $x$'s}

{\scriptsize
\begin{verbatim}
              Analysis of Maximum Likelihood Estimates
 
                                Standard          Wald
 Parameter    DF    Estimate       Error    Chi-Square    Pr > ChiSq

 Intercept     1      9.7539      2.5417       14.7267        0.0001
 shock         1     -3.6739      1.1648        9.9479        0.0016
 malnut        1     -1.2166      0.7282        2.7909        0.0948
 alcohol       1     -3.3549      0.9821       11.6691        0.0006
 age           1     -0.0922      0.0303        9.2353        0.0024
 bowelinf      1     -2.7976      1.1640        5.7767        0.0162

\end{verbatim}
}

\begin{itemize}
\item 
Only marginal one is \verb-malnut-.
\item Test that both \verb-malnut- and \verb-bowelinf- can be removed (suspect not):

{\scriptsize
\begin{verbatim}
                               Wald
             Label       Chi-Square      DF    Pr > ChiSq
             Test 1          6.8302       2        0.0329
\end{verbatim}
}

\item Indeed, not.
\end{itemize}
  
\end{slide}

\begin{slide}{Predictions from model without ``malnut''}

  \begin{itemize}
  \item So fit model without \verb-malnut- and obtain predictions.
  \item A few chosen at random:
{\scriptsize
\begin{verbatim}
 Obs  death  shock  malnut  alcohol  age  bowelinf  _LEVEL_     p

   4    0      0       0       0      26      0        0     0.99858
   1    0      0       0       0      56      0        0     0.97945
   2    0      0       0       0      80      0        0     0.84658

  11    1      0       0       1      66      1        0     0.06871
  32    1      0       0       1      49      0        0     0.78700
\end{verbatim}
}
\item Survival chances pretty good if no risk factors, though decreasing with age.
\item Having more than one risk factor reduces survival chances dramatically.
\item Usually model does a good job of predicting survival, but occasionally someone dies who was predicted to survive.
  \end{itemize}
  
\end{slide}


\begin{slide}{Changing the response category}

  \begin{itemize}
  \item 
In first rats example, got prob of death but maybe wanted prob of living.
\item Change \verb-model- line to this:
\begin{verbatim}
  model survival(event='lived') = dose;
\end{verbatim}
\item Output now includes:
{\scriptsize
\begin{verbatim}
                                     Standard          Wald
  Parameter    DF    Estimate       Error    Chi-Square   Pr > ChiSq
  Intercept     1      1.6841      1.7978        0.8774       0.3489
  dose          1     -0.6736      0.6140        1.2037       0.2726

                 Obs    dose    survival    _LEVEL_      pred
                  1       0      lived       lived     0.84344
                  2       1      died        lived     0.73310
                  3       2      lived       lived     0.58342
                  4       3      lived       lived     0.41658
                  5       4      died        lived     0.26690
                  6       5      died        lived     0.15656

\end{verbatim}
}

  \end{itemize}

  
\end{slide}

\begin{slide}{Testing fit: seroconversion example}

  \begin{itemize}
  \item Seroconversion: body develops specific
    antibodies to microorganisms in blood (as when person gets
    certain disease). 
    \item Seropositive: still
    have antibodies in blood after recovery from
    the disease. 
    \item Malaria survey: ages plus seropositiveness recorded. Data, with variables: age group number, middle of age
    group, \#individuals, \#seropositive:
{\scriptsize
\begin{verbatim}
   1  1.5  123  8
   2  4.0  132  6
   3  7.5  182 18
   4 12.5  140 14
   5 17.5  138 20
   6 25.0  161 39
   7 35.0  133 19
   8 47.0   92 25
   9 60.0   74 44

\end{verbatim}
}
  \end{itemize}
  
\end{slide}

\begin{slide}{Does seropositiveness depend on age?}

Calculate observed pct of seropositives for each age group in DATA step:

\begin{verbatim}
data sero;
  infile "sero.dat";
  input group age n r;
  obspos=r/n;

proc print;
\end{verbatim}

with this result:

{\scriptsize
\begin{verbatim}
                  Obs    group     age     n      r     obspos

                   1       1       1.5    123     8    0.06504
                   2       2       4.0    132     6    0.04545
                   3       3       7.5    182    18    0.09890
                   4       4      12.5    140    14    0.10000
                   5       5      17.5    138    20    0.14493
                   6       6      25.0    161    39    0.24224
                   7       7      35.0    133    19    0.14286
                   8       8      47.0     92    25    0.27174
                   9       9      60.0     74    44    0.59459
\end{verbatim}
}
  
\end{slide}

\begin{slide}{Does a logistic regression fit?}

  \begin{itemize}
  \item 
Prob of being seropositive generally increases with age, but age group 6 has too many seropositives and age group 7 too few. 
\item Fit logistic model anyway, and test for fit.
\item Hosmer-Lemeshow test: 
  \begin{itemize}
  \item null: logistic regression is appropriate
  \item alternative: it is not.
  \end{itemize}
\item Code (note ``events/trials'' syntax and ``lackfit''):

\begin{verbatim}
proc logistic;
  model r/n = age / lackfit;
\end{verbatim}
  \end{itemize}
\end{slide}

\begin{slide}{Hosmer-Lemeshow test output}

{\scriptsize
\begin{verbatim}
           Partition for the Hosmer and Lemeshow Test

                              Event                 Nonevent
 Group       Total    Observed    Expected    Observed    Expected

     1         123           8        8.14         115      114.86
     2         132           6        9.69         126      122.31
     3         182          18       15.43         164      166.57
     4         140          14       14.53         126      125.47
     5         138          20       17.46         118      120.54
     6         161          39       27.11         122      133.89
     7         133          19       31.97         114      101.03
     8          92          25       32.30          67       59.70
     9          74          44       36.38          30       37.62

            Hosmer and Lemeshow Goodness-of-Fit Test

               Chi-Square       DF     Pr > ChiSq
                  21.3185        7         0.0033

\end{verbatim}
}

\end{slide}

\begin{slide}{Interpretation}

\begin{itemize}
\item Actually a chi-squared test based on division of $x$ (age) into groups (here, 9 age groups).
\item P-value 0.0033 small, so logistic regression not appropriate.
\item Maybe age groups 6 and 7 are wrong way around. Assume this (in practice wouldn't, of course)
\item Fit same model again and re-do Hosmer-Lemeshow.
\end{itemize}

  
\end{slide}

\begin{slide}{Output from this analysis}

{\scriptsize
\begin{verbatim}

                             The LOGISTIC Procedure

                   Analysis of Maximum Likelihood Estimates

                                     Standard          Wald
      Parameter    DF    Estimate       Error    Chi-Square    Pr > ChiSq

      Intercept     1     -2.8107      0.1565      322.5387        <.0001
      age           1      0.0476     0.00457      108.4657        <.0001

                    Hosmer and Lemeshow Goodness-of-Fit Test

                       Chi-Square       DF     Pr > ChiSq

                           8.4427        7         0.2952
\end{verbatim}
}

\begin{itemize}
\item No problems with logistic model now.
\item Probability of being seropositive definitely increases with age.
\end{itemize}
  
\end{slide}

\begin{slide}{Predicted probabilities}

{\scriptsize
\begin{verbatim}
 Obs  age  n   r      pobs       pred       lcl        ucl

  1   1.5 123  8    0.06504    0.06069    0.04588    0.07989
  2   4.0 132  6    0.04545    0.06783    0.05227    0.08759
  3   7.5 182 18    0.09890    0.07914    0.06258    0.09961
  4  12.5 140 14    0.10000    0.09830    0.08042    0.11963
  5  17.5 138 20    0.14493    0.12147    0.10230    0.14366
  6  25.0 133 19    0.14286    0.16494    0.14313    0.18934
  7  35.0 161 39    0.24224    0.24115    0.21102    0.27409
  8  47.0  92 25    0.27174    0.35991    0.30883    0.41437
  9  60.0  74 44    0.59459    0.51061    0.43049    0.59018

\end{verbatim}
}

\vspace{3ex}

Plenty of data, so CIs are mostly short. Note clear upward trend in
probabilities.
  
\end{slide}

\begin{slide}{More than 2 response categories}

  \begin{itemize}
  \item With 2 response categories, model the probability of one, and prob of other is one minus that. So doesn't matter which category you model.
  \item With more than 2 categories, have to think more carefully about the categories: are they
    \begin{itemize}
    \item {\em ordered}: you can put them in a natural order (like low, medium, high)
    \item {\em nominal}: ordering the categories doesn't make sense (like red, green, blue).
    \end{itemize}
  \item SAS handles both kinds of response; learn how.
  \end{itemize}
  
\end{slide}

\begin{slide}{Ordinal response: the miners}


  \begin{itemize}
  \item 
Model probability of being in given category {\em or lower}.
\item Example: coal-miners often suffer disease pneumoconiosis. Likelihood of disease believed to be greater 
among miners who have worked longer. 
\item Severity of disease measured on categorical scale: 1 = none, 2
= moderate, 3 = severe.
\item Data are frequencies:
\begin{verbatim}
Exposure None Moderate Severe
   5.8    98      0       0
  15.0    51      2       1
  21.5    34      6       3
  27.5    35      5       8
  33.5    32      10      9
  39.5    23      7       8
  46.0    12      6      10
  51.5     4      2       5
  \end{verbatim}
  
  \end{itemize}
\end{slide}

\begin{slide}{Data setup}

  \begin{itemize}
  \item Set up data file with one frequency on each line, like this: exposure, response category, frequency.
\begin{verbatim}
5.8  1 98
15   1 51
15   2 2
15   3 1
21.5 1 34
\end{verbatim}
\item Don't need to enter zero frequencies.
\item Multiple response categories treated as ordered by default.
\item Make sure ordering in data is the right one! (I use numbers to keep ordering straight.)
  \end{itemize}
\end{slide}

\begin{slide}{Code}

\begin{verbatim}
data miners;
  infile "miners.dat";
  input exposure severity frequency;

proc logistic;
  class severity;
  freq frequency;
  model severity = exposure;
  output out=miners2 pred=pred;

proc print data=miners2;
\end{verbatim}

Note:
\begin{itemize}
\item \verb-class- statement  turns numbers into ordered response
\item \verb=freq= statement ensures frequencies are read as such.
\end{itemize}
  
\end{slide}

\begin{slide}{Output part 1}

{\scriptsize
\begin{verbatim}
                             Model Information

                   Number of Observations Read          22
                   Number of Observations Used          22
                   Sum of Frequencies Read             371
                   Sum of Frequencies Used             371

                               Response Profile
                      Ordered                      Total
                        Value     severity     Frequency
                            1            1           289
                            2            2            38
                            3            3            44

  Probabilities modeled are cumulated over the lower Ordered Values.
\end{verbatim}
}

\vspace{2ex}

22 lines in data file; frequencies indicate 371 miners total. 

\vspace{2ex}

Response profile shows number in each severity category in total.
  
\end{slide}

\begin{slide}{Output part 2}

{\scriptsize
\begin{verbatim}
                 Testing Global Null Hypothesis: BETA=0

         Test                 Chi-Square       DF     Pr > ChiSq
         Likelihood Ratio        88.2432        1         <.0001
         Score                   80.7246        1         <.0001  
         Wald                    64.5206        1         <.0001

                Analysis of Maximum Likelihood Estimates

                                   Standard          Wald
  Parameter      DF    Estimate       Error    Chi-Square    Pr > ChiSq
  Intercept 1     1      3.9559      0.4096       93.2527        <.0001
  Intercept 2     1      4.8691      0.4437      120.4349        <.0001
  exposure        1     -0.0959      0.0119       64.5206        <.0001
\end{verbatim}
}

\vspace{3ex}

Severity of disease definitely depends on exposure. To see how:

\end{slide}

\begin{slide}{Predicted severity probs (edited)}

as they depend on exposure:

{\scriptsize
\begin{verbatim}
       Obs    exposure    severity    frequency    _LEVEL_      pred  
         1       5.8          1           98          1       0.96769 
         2       5.8          1           98          2       0.98678 
         3      15.0          1           51          1       0.92535 
         4      15.0          1           51          2       0.96865 
         9      21.5          1           34          1       0.86920 
        10      21.5          1           34          2       0.94306 
        15      27.5          1           35          1       0.78893 
        16      27.5          1           35          2       0.90306 
        21      33.5          1           32          1       0.67766 
        22      33.5          1           32          2       0.83974 
        27      39.5          1           23          1       0.54181 
        28      39.5          1           23          2       0.74666 
        33      46.0          1           12          1       0.38799 
        34      46.0          1           12          2       0.61241 
        39      51.5          1            4          1       0.27225 
        40      51.5          1            4          2       0.48251 
\end{verbatim}
}
  
\end{slide}

\begin{slide}{Understanding the predicted probs}

  \begin{itemize}
  \item Miner with 5.8 years exposure has prob 0.968 of no disease, and prob 0.987 of moderate disease or lower (and prob 1 of severe disease or lower).
  \item Subtracting: prob of no disease 0.968, moderate disease $0.987-0.968=0.019$, severe disease $1-0.987=0.013$.
  \item Compare with miner with 51.5 years exposure: prob 0.272 of no disease, prob $0.483-0.272=0.211$ of moderate disease, prob $1-0.483=0.517$ of severe disease.
  \item Summary:

    \begin{tabular}{cccc}
      \hline
      Exposure & P(none) & P(moderate) & P(severe)\\
      \hline
      5.8 & 0.968 & 0.019 & 0.013\\
      27.5 & 0.789 & 0.115 & 0.097\\
      51.5 & 0.272 & 0.211 & 0.517\\
      \hline
    \end{tabular}
  \item Miner with more exposure has higher prob of having worse disease.
  \end{itemize}
  
\end{slide}

% mlogit.pdf

\begin{slide}{Unordered responses}

  \begin{itemize}
  \item With unordered (nominal) responses, can use {\em generalized logit}.
  \item Example: 735 people, record age and sex (male 0, female 1), which of 3 brands of some product preferred.
  \item Data in \verb-mlogit.dat- separated by commas.
  \item Tell SAS that sex and brand numbers only distinguish categories.
  \item For predictions, get output data set and inspect.
  \end{itemize}

\end{slide}

\begin{slide}{The code}

\begin{verbatim}
data prefs;
  infile "mlogit.dat" delimiter=",";
  input brand sex age;

proc logistic;
  class brand;
  class sex;
  model brand=sex age / link=glogit;
  output out=mlogit2 pred=pred;

proc print data=mlogit2;
\end{verbatim}

  
\end{slide}

\begin{slide}{Output part 1}

{\scriptsize
\begin{verbatim}
                    Model Information

     Response Variable             brand
     Number of Response Levels     3
     Model                         generalized logit
     Number of Observations Used   735

                     Response Profile
            Ordered                      Total
              Value        brand     Frequency
                  1            1           207
                  2            2           307
                  3            3           221

  Logits modeled use brand=3 as the reference category.
\end{verbatim}
}
  
\end{slide}

\begin{slide}{Output part 2}

{\scriptsize
\begin{verbatim}
           Testing Global Null Hypothesis: BETA=0

   Test                 Chi-Square       DF     Pr > ChiSq
   Likelihood Ratio       185.8502        4         <.0001
   Score                  163.9538        4         <.0001
   Wald                   129.7966        4         <.0001

                  Type 3 Analysis of Effects

                                  Wald
          Effect      DF    Chi-Square    Pr > ChiSq
          sex          2        7.6704        0.0216
          age          2      123.3880        <.0001
\end{verbatim}
}

At least one of sex and age makes a difference to the predicted probs; the bottom table says they both do.
  
\end{slide}

\begin{slide}{Predicted probabilities (a few)}

{\scriptsize
\begin{verbatim}
              Obs    brand    sex    age    _LEVEL_      pred   
                4      1       0      26       1       0.89429  
                5      1       0      26       2       0.09896  
                6      1       0      26       3       0.00674  
               10      1       1      27       1       0.77288  
               11      1       1      27       2       0.20869  
               12      1       1      27       3       0.01843  
             2149      3       0      38       1       0.02598  
             2150      3       0      38       2       0.23855  
             2151      3       0      38       3       0.73547  
             2152      2       1      38       1       0.01623  
             2153      2       1      38       2       0.25162  
             2154      2       1      38       3       0.73215  
\end{verbatim}
}
  
\end{slide}

\begin{slide}{Understanding them}

  \begin{itemize}
  \item Many combinations of age, sex and brand-preferred.
  \item Obs 4, 5 and 6 are for males (sex=0) age 26; prob of preferring brand 1 is 0.894, brand 2 is 0.099, brand 3 is 0.007.
  \item Summarize whole table from previous page:

    \begin{tabular}{ccccc}
      \hline
      Sex & Age & P(prefer 1)& P(prefer 2) & P(prefer 3) \\
      \hline
      Male & 26 & 0.894 & 0.099 & 0.007\\
      Female & 27 & 0.773 & 0.209 & 0.018\\
      Male & 38 & 0.026 & 0.239 & 0.735\\
      Female & 38 & 0.016 & 0.252 & 0.732\\
      \hline
    \end{tabular}
  \item Younger people prefer brand 1, older prefer brand 3.
  \item Females (a little) less likely to prefer brand 1 and more likely to prefer brand 2. (Sex difference {\em is} significant.)
  \end{itemize}
  
\end{slide}

\begin{slide}{Alternative data format}

Summarize all people of same brand preference, same sex, same age on one line of data file with frequency on end:

{\scriptsize
\begin{verbatim}
1 0 24 1
1 0 26 2
1 0 27 4
1 0 28 4
1 0 29 7
1 0 30 3
...
\end{verbatim}
}

Whole data set in 65 lines not 735!
  
\end{slide}

\begin{slide}{Code for alternative data format}

\begin{verbatim}
data prefs;
  infile "mlogit2.dat";
  input brand sex age frequency;

proc logistic;
  class brand;
  class sex;
  freq frequency;
  model brand=sex age / link=glogit;
  output out=mlogit2 pred=pred;
\end{verbatim}

\vspace{3ex}

Add \verb-freq- line in analysis. Output same as before.

  
\end{slide}


\end{document}






